Modern machine learning models trained on losses based on point predictions are prone to be overconfident in their predictions~\citep{guo2017calibration}. 
The Conformal Prediction (CP) framework~\citep{vovk2005algorithmic} provides a mechanism for generating statistically sound post hoc prediction sets (or intervals, in case of continuous outcomes) with coverage guarantees under mild assumptions.
The usual assumption made in CP is that data are exchangeable, i.e, the joint distribution of the data is invariant to permutations of the data points.
The guarantees provided by CP are distribution-free, and can be added post hoc to arbitrary, black-box predictor scores.
This makes CP an ideal candidate for quantifying uncertainty in complex models, such as neural networks.

Network-structured data such as social networks, transportation networks, and biological networks are ubiquitous in modern data science applications.
Graph Neural Networks (GNNs) have been developed to model vector representations of such network-structured data, and have been shown to be effective in a variety of tasks such as node classification, link prediction, and graph classification~\citep{hamilton2020graph, wu2022graph}.
Uncertainty quantification approaches built for independent and identically distributed (iid) data cannot directly be applied to graph data, as the network structure introduces dependencies between the data points.
However, recent work~\citep{clarkson2023distribution,zargarbashi23conformal,huang2024uncertainty} has demonstrated that in certain settings, CP can be applied to graph data to generate statistically sound prediction sets for the node classification task.

Variations of CP include full CP~\citep{vovk2005algorithmic} which has significant computational cost as the score function requires recomputation with replacement for each data point within the calibration set.
 Additionally, cross-conformal prediction~\citep{vovk2015cross}, CV+/Jackknife+~\citep{barber2021predictive} are other variations of CP which are computationally more efficient than full CP, but less efficient than split CP.
 Prior work in CP on graphs has mainly focused on the split CP setting due to its computational efficiency, ease of implementation, and distribution-free guarantees with black-box models. 
 Thus, we focus on split CP in this work.

Owing to the recent nature of this work, there is a lack of consensus for the choice and setup of baselines, splitting of common datasets, and evaluation metrics for methods.
In this work, we aim to analyze the choices made by existing work and understand the trade offs associated with these choices.
In addition, we create a python library which implements different variations of these approaches which would help standardize practices in the evaluation of CP for graph data.
