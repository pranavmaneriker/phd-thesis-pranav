\subsection{Datasets}

\begin{table}
    \centering
    \begin{tabular}{ccccc}
        \toprule
        Dataset & Nodes & Edges & Classes & Features \\
        \midrule
        CiteSeer & 3,327 & 9,228 & 6 & 3,703 \\ 
        Amazon\_Photos & 7,650 & 238,163 & 8 & 745 \\
        Cora & 19,793 & 126,842 & 70 & 8,710 \\
        PubMed & 19,717 & 88,651 & 3 & 500 \\
        Coauthor\_CS &  18,333 & 163,788 & 15 & 6,805 \\
        Coauthor\_Physics & 34,493 & 495,924 & 5 & 8,415 \\
        \bottomrule
    \end{tabular}
    \caption{Summary statistics for Datasets chosen for evaluation.}
    \label{tab:conformal:datasets}
\end{table}


We selected datasets of varying sizes to evaluate the performance of the graph conformal prediction methods.
For the citation datasets, the nodes are publications, and the edges denote citation relationships.
Features are bag-of-words representations of the documents.
The task is to predict the category of each publication.
\textbf{CiteSeer} is a citation network dataset designed for the node classification task, with nodes as publications and edges denoting citation relationships.
\textbf{Amazon\_Photos} is a segment of the Amazon co-purchase graph~\cite{mcauley2015image} where nodes represent goods, edges represent goods frequently bought together, features are bag-of-words representations of product reviews, and the task is to predict the category of each good.
\textbf{Cora} We use CoraFull~\cite{shchur2018pitfalls}, an extended version of the common Cora citation network dataset.
Summary statistics for the datasets are provided in Table~\ref{tab:conformal:datasets}.
The objective is to predict the category of each node (publication).
\textbf{PubMed} is a citation network dataset designed for the node classification task, with nodes as publications and edges denoting citation relationships. The goal is to predict the category of each node (publication).
For all chosen datasets, we used the version provided by the Deep Graph Library~\cite{wang2019dgl}.
\textbf{Coauthor\_CS} and \textbf{Coauthor\_Physics} are co-authorship graphs extracted from the Microsoft Academic Graph and used for KDD Cup 2016. In this dataset, nodes are authors and edges denoting co-authorship relationships. The task is to predict the most active field of study for each author.

To help characterize the behavior of different approaches, we categorize these into sizes based on the number of nodes, with \textbf{Cora} and \textbf{Amazon\_Photos} designated as small (S), \textbf{CiteSeer}, \textbf{PubMed}, and \textbf{Coauthor\_CS} as medium (M), and \textbf{Coauthor\_Physics} as large (L).

\subsection{Metrics}
We evaluate the following metrics for the graph conformal prediction methods:
\begin{itemize}
    \item \textbf{Coverage:} The proportion of test instances for which the true label is contained in the prediction set.
    \item \textbf{Efficiency:} The average size of the prediction set.
    \item \textbf{Label Stratified Coverage:} The mean of coverage for each class. This metric is useful for understanding whether a method is adaptive and has balanced coverage for different classes.
    \item \textbf{Size Stratified Coverage:} The mean of coverage across different sizes of prediction sets. This metric is useful for understanding whether a method is adaptive and does not under/over cover hard/easy samples.
    %\item \textbf{Size Stratified Coverage Violation:} Measures the maximum deviation from the coverage goal across different prediction set sizes. 
    \item \textbf{Singleton Hit Ratio:} The proportion of test instances for which the true label is the only label in the prediction set. This metric measures the frequency with which a method does not require generation of a prediction set having multiple labels.
\end{itemize}

\subsection{Methods}
We discussed the theoretical and empirical tradeoffs of different methods in Section~\ref{chp:graphConformal:sec:conformal_scores_tradeoffs}.
For completeness, we list all the methods that we compare here.
\textbf{Threshold Prediction Sets}~\cite{sadinle2019least}, with two variants, TPS and TPS-Classwise (using class wise thresholds for adapting to class imbalance).
\textbf{Adaptive Prediction Sets}~\cite{romano2020classification} with two variants, APS and APS-Randomized (using the uniform random quantile adjustments).
\textbf{Regularized Adaptive Prediction Sets}~\cite{angelopoulos2021uncertainty}, a variation of APS with a regularization term to ensure that the prediction sets are not too large.
\textbf{Diffused Adaptive Prediction Sets}~\cite{zargarbashi23conformal}, with two variations DAPS and DTPS, which uses a diffusion process over TPS-Classwise.
\textbf{Normalized Adaptive Prediction Sets}~\cite{clarkson2023distribution} with three variations corresponding to the weighing function used.
\textbf{CF-GNN}~\cite{huang2024uncertainty}, a GNN based approach for conformal prediction. We label the original implementation of CFGNN as CFGNN-Original and our improved implementations as CFGNN-APS (using randomized APS as the loss function for training/evaluation) and CFGNN-TPS (using TPS as the loss function for training/evaluation).

\subsection{Notes on Parameter Tuning and Evaluation}

\subsection{Results}