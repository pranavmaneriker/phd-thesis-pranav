The usual tasks of interest in graph data are node classification, link prediction, and graph classification. 
In this work, we focus on the usage of node classification and its extensions to conformal prediction.
\subsection{Node Classification in Graphs}
Consider an attributed homogeneous graph $\gG = (\gV, \gE, \mX)$, where $\gV$ is the set of nodes, $\gE$ is the set of edges and $\mX$ is the set of node attributes.
Let $\gY = \{1, \dots, K\}$ denote set of class labels associated.
For $v \in \gV$, $\vx_v \in \R^d$ denotes the features and $y_v \in \{1, \dots, K\}$ denotes the corresponding class label.
\pmcomment{Exchangeability for transductive, and potential for inductive}

\subsection{Conformal Prediction for Node Classification}
Let $f: \gX \to \R_K$ denote the class wise scores associated with a  model trained on a separate split of the data ($\gD_{\text{train}}$).
For example, these could be the pre-final output layer of a neural network (either before or after softmax normalization).
and a trained model with classwise prediction the goal is to learn a model $\pi: \gX \to \Delta_K$, where $\Delta_K$ is the $K$-dimensional probability simplex.
