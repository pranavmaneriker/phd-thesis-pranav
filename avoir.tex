\chapter{Auditing Fairness Online through Iterative Refinement}
\label{chp:avoir}

Machine learning algorithms are increasingly being deployed for high-stakes scenarios. 
A sizeable proportion of currently deployed models make their decisions in a black-box manner. 
Such decision-making procedures are susceptible to intrinsic biases, which has led to a call for accountability in deployed decision systems.
In this work, we focus on user-specified accountability of decision-making processes of black box systems.
Previous work has formulated this problem as run time fairness monitoring over decision functions.
However, formulating appropriate specifications for situation-appropriate fairness metrics is challenging.
We construct \AVOIRmethodname{}, an automated inference-based optimization system that improves bounds for and generalizes prior work across a wide range of fairness metrics.
\AVOIRmethodname{} offers an interactive and iterative process for exploring fairness violations aligned with governance and regulatory requirements.
Our bounds improve over previous probabilistic guarantees for such fairness grammars in online settings.
We also construct a novel visualization mechanism that can be used to investigate the context of reported fairness violations and guide users toward meaningful and compliant fairness specifications. 
We then conduct case studies with fairness metrics on three different datasets and demonstrate how the visualization and improved optimization can detect fairness violations more efficiently and alleviate the issues with faulty fairness metric design. 


\section{Introduction}
%\pmcomment{Show that you can do this on different types of datasets. Verifair only runs on simulated datasets - compared to our results on a broad class of datatypes.}

%\pmcomment{One pass over introduction completed}

%\pmcomment{Currently working on merging introduction and related work}

The use of advanced analytics and artificial intelligence (AI), along with its many benefits, poses important threats to individuals and the broader society at large.
%\pmcomment{SRINI - I think it is ok - its been out for a couple of years and has a few cites already. Pranav - We may need another reference in addition to this for anonymity}
\cite{hirsch20corporate} identify invasion of privacy; manipulation of vulnerabilities;  bias against protected classes; increased power imbalances;  error; opacity and procedural unfairness; displacement of labor;  pressure to conform, and intentional and harmful use as some of the key areas of concern.
A core part of the solution %space
to mitigate such risks is the need to make organizations accountable and ensure that the data they leverage and the models they build and use 
are both inclusive of marginalized groups and resilient against societal bias.
Deployed AI and analytic systems are complex multi-step processes that can produce several sources of risk at each step.
%\begin{leftbar}
At each of these stages, determining accountability in the decision-making in AI processes requires a determination of who is accountable, for what, to whom, and under what circumstances~\citep{nissenbaum1996accountability,cooper2022accountability}. 
\todo{Contextualizing wrt Nissenbaum}
%\marginpar[]{}
A more comprehensive overview of the mechanisms that can support accountability with respect to the different stages of design of a machine learning system can be found in the work of \citet{cooper2022accountability}.
%\end{leftbar}
We center our analysis on the sub-problem of auditing barriers towards investigating claims surrounding mathematical guarantees of automated decision making processes.
Governments across the world are wrestling with the implementation of auditing regulation and practices for increasing the accountability of decision processes.
Recent examples include the New York City auditing requirements for AI hiring tools~\citep{vanderford2022nybiaslaw}, European data regulation (GDPR~\citeyear{GDPR}), accountability bills~\citeyear{congress2019hr,algtransparency2022} and judicial reports~\citeyear{justice2018free}.
%In an evolving environment, organizations also aim to augment their understanding of algorithmic audits.
These societal forces have led to the emergence of checklists~\citep{mitchell2019model,sokol2020explainability}, metrics of fairness~\citep{verma2018fairness}, and recently, algorithms and systems that observe and audits the behavior of %advanced analytics and
AI algorithms.  %We note that 
Such ideas date back to the 1950s~\citep{moore1956gedanken} % for black box automata), 
but research has largely been sporadic until very recently with the widespread use of AI-based decision making giving rise to the vision of algorithmic auditing~\citep{clavell2020auditing}.
We present a framework for {\it Auditing and Verifying fairness Online through Interactive Refinement}~(\AVOIRmethodname{})
\footnote{AVOIR in French means ``to have'' and this acronym reflects both our aspirational goal to achieve fairness in advanced analytics and AI but also reflects what is currently verifiable given a dataset, a model and a fairness specification.}.
\AVOIRmethodname{} builds upon the ideas on distributional probabilistic fairness guarantees~\citep{albarghouthi2019fairness,bastani2019probabilistic}, generalizing them to real-world data. 
An overview of \AVOIRmethodname{} is provided in Figure~\ref{fig:framework}.
%Delivering on this vision would allow organizations and regulators to gain some understanding on whether the algorithmic decision-making models they are using are verifiably meeting regulatory standards.
%Such ideas are  essential to building trust in service platforms and facilitate platform regulation. 
%Importantly such ideas are needed both at the data and  algorithmic  level - mirroring recent regulatory advances.
%Despite their ubiquitous use, biases from the labeled data or algorithm design can arise and raise concerns about machine learning decisions.
%Further, some applications require decision-making processes to follow regulatory frameworks.

\begin{figure}[ht]
    \centering
    \includegraphics[width=0.7\linewidth]{avoir/images/Framework.pdf}
    \caption{Shaded nodes describe our contributions in the \AVOIRmethodname{} framework.}
    \label{fig:framework}
\end{figure}

\subsection{Preliminaries}
Machine learning testing~\citep{zhang2020mltesting} is an avenue that can be used to expose undesired behavior and improve the trustworthiness of machine learning systems.
%Testing in software systems relies on assertions over the behavior of the functions being tested.
Fairness criteria quantify the relationship between the outcome metric across multiple subgroups or similar individuals among the population.
Formal definitions of fairness focus on observational criteria, i.e., those that can be written down as a probability statement involving the joint distribution of the features, sensitive attributes, decision making function, and actual outcome. %from https://fairmlclass.github.io/4.html#/2
Our framework, \AVOIRmethodname{}, supports implementing a large range of group fairness criteria, including demographic parity~\citep{calders2009building}, equal opportunity~\citep{hardt2016equality}, disparate mistreatment~\citep{zafar2017fairness}, and various combinations of these criteria. 
%We give an example of a fairness metric and the corresponding fairness specification here.
%For example, a decision making function that selects amongst applicants to hire for a specific position may have different hiring rates for majority and minority population groups. 
As an example, suppose $r \in \{ 0, 1\}$ denotes the return value of a binary decision function (say, candidate selection for a job), and $s$ is an indicator denoting whether a candidate belongs to a minority population.
The 80\%-rule for disparate impact~\citep{eeoc1979,feldman2015certifying} is a fairness criterion which states that
\begin{align*}
    \frac{\Pr[r=1| s]}{\Pr[r=1| \neg s]} \geq 0.8 
\end{align*}
\todo{Section to introduce the terms.}
%Decision making/scoring functions aim to optimize for a specific metric such as accuracy, precision, F1 score etc. 
%Fairness criteria, on the other hand, quantify the relationship between the outcome metric across multiple subgroups or similar individuals among the population. 
When implemented in the \AVOIRmethodname{} DSL grammar, the above 80\%-rule would be the specification \lstinline{E[r|S==s] / E[r|S!= s] >= 0.8}.
Here, the term \lstinline{E[r|S!=s]/E[r|S == s]} is a \textit{subexpression} of the specification.
The smallest units involving an expectation (eg., \lstinline{E[r|S!=s]}) are denoted as an \textit{elementary subexpressions}.
Our algorithm works by using adaptive concentration sets~\citep{zhao2016adaptive,howard2021time} to build estimates for \textit{elementary subexpressions}, and then deriving the estimates for expressions that combine them.
We aim to derive statistical guarantees about fairness criteria based on estimates from observed outputs.
For example, let $X$ be an observed Bernoulli r.v\footnote{random variable}, then an assertion $\phi_X = (\BarE[X], \epsilon, \delta)$ over $X$, corresponds to an estimate satisfying
\begin{equation}\label{eqn:eps-delta-defn}
  \phi_X \equiv \Pr[|\E[X] - \BarE[X]| \geq \epsilon] \leq \delta  
\end{equation}
where $\BarE[X]$ denotes an empirical estimate of $E[X]$.
We then use assertions $\phi_X, \phi_Y$ to assert claims for expressions involving $X, Y$.
For example, for the 80\%-rule, assertions over $X/Y$.
A specification involves either a comparison of expressions with constants (eg., $X/Y > 0.8$), or a combination of multiple such comparisons. 
Such a specification may be True ($T$) or False ($F$) with some probability.
For a given specification $\psi$, we denote the claim that $P[\psi = F] \geq 1 - \delta$ as $\psi: (F, \delta)$, where $\delta$ denotes the failure probability of a guarantee. 
Given a stream of (observations, outcomes from the decision functions), and a specified threshold probability $\delta$, we will continue to refine the estimate for a given specification until we reach the threshold.
We focus on fairness criteria that can be expressed using Bernoulli r.v. as it allows the simplification of probabilities into expectation, as $\Pr[r = 1] = \E[r]$.
Specifications involving variables that take more than two values can be implemented using transformations and boolean operators (examples in Appendix~\ref{sec:appendix:additional-metrics}).
%\pmcomment{Discussion of Individual vs Group fairness here?}
%Section~\ref{sec:casestudy} contains descriptive analyses and case studies of real world applications of \AVOIRmethodname{} on commonly used fairness criteria.
%\pmcomment{more about fairness criteria group fairness etc?}

%\subsection{Fairness Specifications}
%The method will terminate when the specification can be asserted as being either True or False with a likelihood $> 1 - \delta$.
%\pmcomment{We can use $\HatMu_X$ instead of $\BarE[X]$ - does it look cleaner?}
\section{Related Work}
\label{sec:related}

There are a plethora of fairness criteria, and subtle changes in their definition can change the implications on decision-making~\cite{castelnovo2021zoo}.
Practitioners need support when selecting, designing, and guaranteeing fairness for deployed machine learning algorithms.
Prior work on fairness has helped develop nuanced notions and algorithms to help train more `fair' machine learning models.
These include group fairness measures such as inter alia, minimizing disparate impact~\citep{calders2009building,feldman2015certifying}, maximizing the equality of opportunity~\citep{hardt2016equality}
In contrast with group fairness notions, causal notions of fairness~\cite{kusner2017counterfactual} and individualized notions of fairness~\cite{dwork2012fairness} provide alternative statistical mechanisms for understanding discriminatory behaviors of automated decision systems.
\citet{thomas2019preventing} proposed the Seldonian Framework as a generic mechanism for model users to design algorithms that help train machine learning models that can regulate them against undesirable behaviors.
\citet{yan2022active} propose a query-efficient framework to audit an unknown function chosen from a known hypothesis class of decision-making functions.

We focus on the problem of detecting and diagnosing whether systems designed under any framework follow any prescribed regulatory constraints supported within the grammar of \AVOIRmethodname{}.
That is, we are agnostic to the framework; instead, we are interested in testing the adherence of models to specified criteria.
We use a probabilistic framework to verify this behavior.
Alternative frameworks such as the AI Fairness 360~\citep{bellamy2019AI}  provide mechanisms to quantify fairness uncertainty, though they are restricted to pre-supported metrics.
Uncertainty quantification~\citep{ghosh2021uncertainty,ginart2022mldemon} is an alternative mechanism to provide adaptive guarantees. 
However, existing work is designed for commonly used outcome metrics, such as accuracy and F1-score, rather than for fairness metrics. 
Justicia~\citep{ghosh2021justicia} optimizes uncertainty for fairness metrics estimates using stochastic SAT solvers but can only be applied to a limited class of tree-based classification algorithms.

Machine learning testing~\citep{mltesting} is an avenue that can expose undesired behavior and improve the trustworthiness of machine learning systems.
Prior work on fairness testing is most closely related to \AVOIRmethodname{}.
Fairness testing~\citep{galhotra2017fairness} provides a notion of causal fairness and generates tests to check the fairness of a given decision-making procedure.
Given a specific definition of fairness, Fairtest~\citep{fairtest} and Verifair (VF)~\citep{bastani2019probabilistic} build a comprehensive framework for investigating fairness in data-driven pipelines. 
%By auditing the system's fairness, a more balanced rule can be generated for future use.
Fairness-aware Programming (FP) \citep{albarghouthi2019fairness} combined the two demands of machine learning testing and fairness auditing to make fairness a first-class concern in programming. 
Fairness-aware programming applies a runtime monitoring system for a decision-making procedure with respect to an initially stated fairness specification.
The overall failure probability of an assertion is computed as the sum of the failure probabilities of each constituting sub-expression (using the union bound).
FP does not provide any specific mechanism for splitting uncertainty, and Verifair splits it equally across all constituent \textit{elementary subexpressions}.
Thus, assertion bounds for subexpressions in both FP and VF are split inefficiently compared to \AVOIRmethodname{}. 

\subsection{\AVOIRmethodname{}: Key Contributions}
\label{sec:contributions}

We now summarize our contributions vis-à-vis FP and Verifair.
(1) We build up \AVOIRmethodname{} in the framework of \textbf{confidence sets}~\citep{howard2021time} which enables the \textbf{adaptive optimization} of $\delta$ across subexpressions.
Note that FP provides examples with equal splits across two terms though it makes no specific prescription of splits.
Verifair splits uncertainty equally across all elementary subexpressions.
(2) The confidence sets framework allows us to move away from assuming a known data distribution or alternatively, fitting a density estimator over the population prior to fairness testing, required in Verifair.
(3) We augment the \textbf{bound propagation rules} to facilitate the online optimization process and allow propagation of constraints along with assertions at each iteration.
\todo{New contributions section.}
(4) We build an \textbf{inference engine} that supports automated inference of propagation rules for wide range of metrics.
In Section~\ref{sec:casestudy}, we provide examples of inference over specifications involving over two subexpressions, which are not possible without extending the implementations provided by previous work. 
As a baseline, we also implement bound inference rules from Verifair (denoted \AVOIRmethodname{}-VF).
(5) We support \textbf{interactive diagnosis} of fairness specification violations using visual cues associated with convergence of subexpressions.
We demonstrate the use of these cues to help drive the design of specifications in Section~\ref{sec:casestudy:adult}, which show how a user may have audited their original claim and refined mathematical bounds.
%More details about the differences referenced above are also available in Appendix~\ref{sec:appendix:contributions}.
\section{\AVOIRmethodname{} Framework}
\label{sec:theoretical}
%\pmcomment{1) Language spec, 2) inference of confidence rules 3) Define confidence sets/bounds, 4) Describe our algorithm, 5) show that our setting is a confidence set, 6) Prove that our method is better 7) Concrete Example}
\subsection{Language Specification}
\label{sec:theoretical:specification}

 
%Following~\cite{albarghouthi2019fairness}, for demonstrating its use, we build our framework as a library for specifying fairness criteria as decorators over python functions. 
We describe \AVOIRmethodname{}'s Domain Specific Language (DSL) used for specifying fairness metrics.
Concrete examples of implemented specifications in \AVOIRmethodname{}'s DSL are provided in Section~\ref{sec:casestudy}.
We focus on binary decision making functions; their outputs can be characterized by Bernoulli r.v.s.
Note that for such a Bernoulli r.v. $X$, $\E[X] = \Pr[X = 1]$ and hereafter, these are used interchangeably. 
Fairness specifications are implemented as decorators over decision functions.
Consider a decision function $f: X \rightarrow \{0, 1\}$, where $X = (X_1, \dots, X_k)$ denotes a real-valued input vector.  We use $R = f(X)$ to simplify the remainder of the definitions. 
\begin{itemize}
    \item  To support expressions beyond those that produce binary outputs, we use the grammar to construct Bernoulli r.vs. For example, a $\nu$-threshold based real-valued output, $R' = (R > \nu)$ and a multi-class output, for class $j$,  $R' = (R == j)$ correspond to Bernoulli r.vs.
\todo{Expanded definitions}
    \item  Expressions involving $R$ and $X_i$ act as the arguments \lstinline{<E>} to construct an \lstinline{<ETerm>}. For example  $\E[R > 0 | X_1 + X_2 > a]$ is an \textit{elementary subexpression} and an \lstinline{<ETerm>}
\end{itemize}
$c$ terms represent constant real values, used, for example, as bounds for expressions.
The grammar provided in Figure~\ref{fig:grammar} can be then used to construct various group fairness criteria.
%Nonparametric confidence sequences~\cite{howard2021time} can be used to extend these results to other r.vs.
%The grammar can be described using a Dmain Specific Language (DSL).
%\paragraph{Grammar for Specification DSL} 
%We start with the grammar used by prior work and enhance the grammar to simplify the expressions used for common fairness specifications. 
%Figure~\ref{fig:grammar} describes the full grammar. 
%\subsubsection{Modified Grammar}
We modified the grammar from prior work to include two additional operations. 
First, we added a \texttt{given} argument to the expectation term, which allows a user to specify conditional probabilities directly, in contrast to specifying it as a ratio of joint/marginal probabilities. 
 %In \citet{albarghouthi2019fairness}, conditional probabilities need to be specified as a ratio of the joint probability divided by the marginal probability of the conditional, i.e., $\E(A|B)$
 \begin{align*}
     \frac{\E(A \vee (B=b))}{\E(B=b)} \rightarrow \E(A, \mathtt{given}=(B=b))
 \end{align*}
 which is used to represent $\E[A|B=b]$, simplifying expressions used for group fairness specification.
Additionally, we add binary comparison operators $<, >, ==, !=$, which further simplifies the process of writing specifications. %\citet{albarghouthi2019fairness} only consider the $>$ operator as a part of their grammar. Thus, we build a more expressive grammar.

\subsection{Propagating Bounds}
\label{sec:theoretical:propagation}
%\paragraph{Inference and Optimization}
Generating the bounds for a specification requires propagating guarantees from elementary subexpressions.
Assuming that observed values for each \texttt{<E>} correspond to an underlying random variable $X$,
a probabilistic guarantee $\phi_X$ consists of an empirical estimate $\BarE[X]$, a concentration bound $\epsilon_X$, and a failure probability $\delta_X$, such that $\Pr[|\E[X] - \BarE[X]|\geq \epsilon_X] \leq \delta_X$.
We refer to expressions of this form as \textit{elementary} subexpressions.
A fairness specification will typically consist of multiple such elementary expressions, denoted as {\it compound} expressions.
For compound expressions, we must infer the implied guarantees that can be provided, with corresponding constraints.
%In addition, guarantees may require certain constraints to be satisfied.
Each inference rule corresponds to a derivation in the DSL grammar.
Inference rules have preconditions and postconditions that follow the general expression
\begin{align*}
 \frac{\bigcup \left\{r | r \in \{\phi, \psi, C \}\right\}}{\bigcup \left\{s | s \in \{ \phi, \psi, C \} \right\} }
\end{align*}
where $\phi$ denotes a claim for a subexpression, $\psi$ for a \verb|<spec>|, $\BarE$ and $\epsilon$ are the mean and concentration terms associated with a subexpression claim,  $C$ denotes a constraint. 
\todo{Updated the language to show the assumptions} 
For example, consider starting with the assumptions $X: (\BarE[X], \epsilon_X, \delta_X)$, $Y: (\BarE[Y], \epsilon_Y, \delta_Y)$.
Then we have
\begin{align*}
    |\E[X] \pm  \E[Y] - (\BarE[X] \pm \BarE[Y])| &= |(\E[X] - \BarE[X]) \pm  (\E[Y] - \BarE[Y])| \\
                                                  & \leq |\E[X] - \BarE[X]| + |\E[Y] - \BarE[Y]|\\
                                                  & \leq \epsilon_X + \epsilon_Y
\end{align*}
i.e., we can derive $X \pm Y: \left(\BarE[X] \pm \BarE[Y], \epsilon_X + \epsilon_Y, \delta_X + \delta_Y\right)$.
Some derivations also lead to rules that require constraints.
For instance, assume $X: (\BarE[X], \epsilon_X, \delta_X), \BarE[X] > c$. 
Then we have $\Pr[X < \BarE[X] - \epsilon_X] > 1 - \delta$
If we add the constraint that $\BarE[X] - \epsilon_X \geq c$, we have $\Pr[X < c] > 1 - \delta$, thus, 
\begin{align*}
    X: (\BarE[X], \epsilon_X, \delta_X) \implies \psi \equiv X > c: (T, \delta_X) \\
    \text{under the constraint } \{ \BarE[X] - \epsilon_X \geq c\}
\end{align*}
The full set of inference rules required for the DSL is provided in the appendix (\Figref{fig:inference}).
The implementation in \AVOIRmethodname{} follows these rules but can be extended to other rule inference templates that support the DSL.
We note that these rules extend the ones implemented by VeriFair (VF)\footnote{Verifair}~\citep{bastani2019probabilistic} with constraints that enable the optimizations required in \AVOIRmethodname{} (see Appendix~\ref{sec:appendix:inference-rules}). 
%\pmcomment{SP: Point to appendix?}

%\pmcomment{Keep constraint rules}




\subsection{Optimizing Bounds}
\label{sec:theoretical:optimization}


\subsubsection{\AVOIRmethodname{} Algorithm}

%\pmcomment{Move to algorithm environment}
%\pmcomment{reference MLP and interior point methods}
The pseudocode for the optimization procedure in \AVOIRmethodname{} is described in  the appendix (Algorithm~\ref{alg:method}).
The input to the algorithm is the reporting threshold probability $\Delta$ and a specification $\psi$.
We then infer a symbolic optimization problem is inferred corresponding to the failure probabilities and constraints derived from concentration bounds.
At each step, the \texttt{OBSERVE(X)} function is called with new observation of every \textit{elementary} subexpression and observed output.
The running mean and counts of observations are updated.
The final optimization problem \texttt{OPT} corresponding to each specification is a nonlinear constrained optimization problem.
We use the COIN-OR implementation of IPOPT~\citep{wachter2006implementation}, accessed though the Pyomo~\citep{hart2011pyomo} interface to solve this problem at each step.
If a solution is successfully found for \texttt{OPT}, the algorithm terminates, with the estimate for the specification having reached the required threshold.
If no solution is found, the estimates continue to be updated with $\delta_i = \Delta$ for each \textit{elementary} subexpression.
The main intuition behind the algorithm is to create a confidence sequence corresponding to the estimates at each time step.
The \texttt{OPT} corresponding to a specification:
\begin{align}
    \label{eq:optimization}
    \begin{split}
        &\min_{\delta_i} \sum_{i=1}^{n}\delta_i  \\
        \text{s.t. } &g_k(\delta_{1, \dots, n}, \BarE[X_1], \dots, \BarE[X_n]) \leq \epsilon_k\\
        & 0 \leq \delta_i \leq 1
    \end{split}
\end{align}
where $g_k$ and $\epsilon_k$ are the functions/bounds derived using the transformations carried out through the DSL inference rules (further details in Appendix~\ref{sec:appendix:inferrence-rules:opt}).


\begin{definition}
For $\delta \in (0, 1)$, a $(1-\delta)$ \textit{confidence sequence} is a sequence of confidence sets, usually intervals  $(\rm{CI}_t)_{t=1}^\infty,$, say $\rm{CI}_t \eqdef (L_t, R_t) \subseteq \sR$ satisfying a uniform convergence guarantee. After observing the $t$th unit, we calculate an updated confidence set $\rm{CI}_t$ for an unknown quantity of interest $\theta_t$ with the coverage property $\Pr(\forall t \geq 1, \theta_t: \theta_t \in \rm{CI}_t) \geq 1 - \delta$~\citep{howard2021time}.
\end{definition}

In this paper, we focus on the mean of r.v.s $\E[X]$ that constitute estimates for \textit{elementary} subexpressions as the quantities of interest. 
We use adaptive concentration inequalities to construct these confidence sequences.
Any adaptive concentration inequality that can be applied to a r.v. $X \in \{0, 1\}$ such that 
\begin{equation}
    \Pr[|\BarE_t[X] - \E[X]| \geq \epsilon(t, \delta)] \leq \delta
    \label{eq:adaptive-conc:general}
\end{equation}
can be used in \AVOIRmethodname{}. 
Here, $\BarE_t[X]$ denotes the empirical estimate of $\E[X]$ after the $t^{\rm{th}}$ observation.
For the purpose of comparison with previous work (eg., VF), we use the Adaptive Hoeffding Inequality~\citep{zhao2016adaptive}, which will be referred to as $\rm{AIN}$ hereafter.
%\pmcomment{SP: refer to $\rm{AIN}$ above but $\rm{AIN}$ below. Check and unify.}

\begin{theorem}
The sequence of estimates generated by \AVOIRmethodname{} form a confidence set.
\label{thm:conf-seq}
\end{theorem}
The proof follows from the fact that \AVOIRmethodname{} always estimates using a failure probability higher than that which is provided by $\rm{AIN}$, and hence applying a union bound ensures that the estimates are a confidence set. 
The full proof is provided in Appendix~\ref{sec:appendix:confseq}.

\begin{corollary}
The estimates for the overall specification $\psi$ form a confidence sequence converging to $\psi: (b, \Delta), b \in \{T, F\}$.
\end{corollary}
\begin{proof}
We initialize the main specification with the required failure probability $\Delta$. 
The termination condition requires $\sum \delta_i \leq \Delta$.
From Theorem~\ref{thm:conf-seq} we can infer that the confidence sequence corresponding to the termination achieves the required threshold $\Delta$, and therefore, is valid.
\end{proof}


\subsubsection{Improvements over Baseline}
In all prior work~\citep{albarghouthi2017fairsquare,albarghouthi2019fairness,bastani2019probabilistic}, $\delta_i$ for each \textit{elementary} subexpressions is set to $\Delta/n$, where $n$ is the number such term in the specification.
\todo{Introduce $A_\delta$ }
This simplification is carried out using the assumption $A_\delta \eqdef \delta_i = \delta_j \forall i, j$ for all \textit{elementary} subexpressions.
As we do not make this assumption, we can prove the following key theorem.

\begin{definition}
We define the specification stopping time $\gT$ for a confidence sequence as the smallest time $t$ such that, given a threshold $\Delta$ and a specification $\psi$, we can terminate any inference algorithm to claim that  $\Pr[\forall t \geq 1, \psi_t = \widehat{\psi}_\gT] \geq 1 - \Delta$, where $\widehat{\psi}_{\gT}$ is the estimate of $\psi$ at time $\gT$.
\end{definition}

\begin{theorem}
\label{theorem:better-stopping}
Given a threshold probability $\Delta$ for a specification $\psi$, let the stopping time for \AVOIRmethodname{} be $\gT$ and stopping time with the $A_\delta$ assumption be $\gT^+$. Then $\gT \leq \gT^+$
\end{theorem}
See Appendix~\ref{sec:appendix:optimality} for the proof.

%\pmcomment{Need to make this argument rigorous}


%\pmcomment{Write this proof replacing AH/Verifair with arbitrary $\rm{IN}$}

\paragraph{Concrete Example}
%\label{sec:theoretical:improvements:concrete}
%\pmcomment{Prior work makes assumptions of equality of unceertainty thresholds; as we do not make this assumption, which allows us to optimize over prior owrk.}
Consider a Bernoulli r.v $R$ corresponding to the output of a binary decision function, with $s$ being an indicator of class membership. 
Let $X = r \vee s$ and $Y = r \vee \neg s$ be r.vs corresponding to a positive decision for the majority and minority classes, respectively. 
Suppose we aim to estimate $\psi \eqdef X - Y < \epsilon_T$

\begin{figure}[ht]
\begin{subfigure}[b]{0.45\linewidth}
\resizebox{\linewidth}{!}{
    \begin{tikzpicture}
        \centering
        \begin{axis}[
                %xtick = \empty,    ytick = \empty,
                grid=both,
                xlabel = {$\delta_X$},
                %x label style = {at={(1,0)},anchor=west},
                ylabel = {$\delta_Y$},
                %y label style = {at={(0,1)},rotate=-90,anchor=south},
                y label style = {rotate=-90},
                axis lines=left,
                xmin=0.03, xmax=0.06,
                ymin=0.04, ymax=0.07,
                label style={font=\Large}
            ]
            \addplot[color=cb-rose,thick,smooth,domain=0:0.1]{x};
            \addlegendentry{$\delta_X = \delta_Y$}
            
            
            \addplot[name path=delta, color=cb-lilac,thick,smooth,domain=0:0.1,forget plot]{0.1-x};
            \path[name path=axis2] (0,0) -- (0.1,0);
            \addplot [
                thick,
                color=cb-lilac,
                fill=cb-lilac, 
                fill opacity=0.05,
                draw=cb-lilac
            ]
            fill between[
                of=axis2 and delta, soft clip={domain=0:0.1}
            ];
            \addlegendentry{$\delta_X + \delta_Y \leq 0.1$};
            
            
            \addplot[name path=constraint,color=cb-brown,smooth,thick,-,domain=0:0.1,forget plot] {24/exp((9/5)*(((0.2 - sqrt(2.6146 + 5*ln(24/x)/9)/(5*sqrt(62)))^2) * 310 - 2.46838))};
            \path[name path=axis3] (0,0.1) -- (1,0.1);
            \addplot [
                thick,
                color=cb-brown,
                fill=cb-brown, 
                fill opacity=0.05,
                draw=cb-brown
            ]
            fill between[
                of=constraint and axis3,
                soft clip={domain=0:0.1}
            ];
            \addlegendentry{$\epsilon_X + \epsilon_Y \leq 0.2$}
        \end{axis}
    \end{tikzpicture}}
        \caption{No solution exists with additional constraint $A_\delta: \delta_X = \delta_Y = \Delta/2$ - common assumption in prior work.}
        \label{fig:theoretical-example}
    \end{subfigure}
    \hfill
    \begin{subfigure}[b]{0.45\linewidth}
        \centering
        \includegraphics[width=\linewidth]{avoir/images/ratemyprofs.png}
        \caption{Bounds for first half of a gender-fairness specification generated by \AVOIRmethodname{}-OB and AVOIR-VF.}
        \label{fig:casestudy:rmp}
    \end{subfigure}
    \caption{\figleft{} \AVOIRmethodname{} finds a solution for a \textit{theoretical} scenario with $\delta_X + \delta_Y \leq \Delta$ under constraint $\epsilon_X + \epsilon_Y \leq \epsilon_T$ \figright{} For \textit{RateMyProfs}, a real-world dataset, the vertical lines show the step at which the methods can provide a guarantee of failure for the upper bounds with $\Delta <= 0.05$.}
    \label{fig:real-and-theoretical-examples}
\end{figure}


We demonstrate the improvements possible using our approach by instantiating this example with data. 
Suppose we want the upper bound of the failure probability $\Delta = 0.1$ for the specification.
Consider a set of observations such that $\BarE[X] = 0.8, n_X = 1550$ and $\BarE[Y] = 0.5, n_Y = 310$.
Figure~\ref{fig:theoretical-example} shows that no solution is feasible for the optimization problem with $A_\delta$.
However, \AVOIRmethodname{} can find a solution.
For the optimal solution, $\delta_2 \approx 2.35\delta_1$, which aligns with our intuition from section~\ref{sec:related} about allocating higher failure probability to terms with the majority of observations. 
The optimization problem inferred by \AVOIRmethodname{}:
\begin{align}
    \begin{split}
        &\min\limits_{\delta_X, \delta_Y}{\delta_X + \delta_Y} \\
        \text{s.t.  } &\epsilon_X + \epsilon_Y \leq \BarE[X] - \BarE[Y] - \epsilon_T \\
        &0\leq \delta_{X,Y} \leq 1
    \end{split}
\end{align}

%\pmcomment{Please point reader to location in appendix on proof.}


 

\begin{comment}
$\Pr[X] - \Pr[Y]  < \epsilon_T$. 
As $\Pr[X] = \E[X]$ for a Bernoulli r.v., this can be simplified as:
\begin{align*}
    \Pr[\Pr[X] - \Pr[Y] < \epsilon_T] &= \Pr[\E[X] - \E[Y] < \epsilon_T] \\
                                  &= 1 - \Pr[\E[X] - \E[Y] \geq \epsilon_T] \\
\end{align*}

Suppose $\BarE[X]_{(\epsilon_X, \delta_X)}$ and $\BarE[Y]_{(\epsilon_Y, \delta_Y)}$ be statistical guarantees derived for $X$ and $Y$ respectively. 
From the inference rules in Figure~\ref{fig:inference}, we have 
\begin{align*}
    \Pr[|(\E[X]-\E[Y]) - (\BarE[X] - \BarE[Y])| \geq \epsilon_X + \epsilon_Y] &\leq \delta_X + \delta_Y\\
    \implies \Pr[(\E[X]-\E[Y]) \geq \BarE[X] - \BarE[Y] -  (\epsilon_X + \epsilon_Y)] &\leq \delta_X + \delta_Y \\
    \implies \Pr[(\E[X]-\E[Y]) \geq \epsilon_T] &\leq \delta_X + \delta_Y
\end{align*}
where the last inequality follows if $\BarE[X] - \BarE[Y] - (\epsilon_X + \epsilon_Y) \geq \epsilon_T$
\end{comment}


%\pmcomment{TODO: replace the screenshot with a vector graphic from tikz}

\subsection{Visualization for Interactive Refinement}
Using our specification framework as a backend, we built an interactive application for analysis and refinement of specifications provided in our grammar.
Specifically, fairness specifications can be naturally parsed into a tree because of the structure of the grammar.
Each node of the tree represents some sub-expression in the syntax tree of the overall specification.
These nodes allow a user of \AVOIRmethodname{} to interactively audit and tune the specification definition.
To create the visualization, we use Vega~\citep{satyanarayan2015reactive}, a declarative JSON-based visualization grammar.
%To provide the evaluation plots, we need the evaluation values and the observations that they occurred at.
We log the estimates during runs of \AVOIRmethodname{} and then output the grammar in a tabular JSON-format that contains a row for each grammar element and its associated evaluations.
This tabular data is used by our Vega specification to produce the visualizations.
By selecting one of the nodes in the syntax tree, a user can see a plot of the evaluation values associated with the selected grammar element.
This allows for comparison of multiple grammar elements.
The ability to analyze and compare these evaluation values provides context surrounding specification violations, and assists the user in interacting with and deciding how to refine a specification
We provide a detailed example of how these interactions can help \AVOIRmethodname{} users choose an appropriate fairness metric in Section~\ref{sec:casestudy}.
%Given a user provided machine learning model, dataset, and specification the application simulates a stream of observations to the provided model.
%Following the simulation, a visualization is provided that represents the specification as a syntax tree where each node of the tree corresponds to an element of our grammar.
%Figures \ref{fig:casestudy:boston} and \ref{fig:casestudy:adult} show the visualization.
%Note that for each observation made by our machine learning model, the specification is evaluated to check for violations.
%Each grammar element that makes up the specification is evaluated as well, and thus each grammar element is associated with the value it evaluates to for a given observation.
%For the top level specification, \texttt{<spec>}, there is a boolean value associated with each observation, whereas an expectation term, \texttt{<ETerm>}, is associated with a real value.
%We call these plots evaluation plots and two can be observed at a time (see the plots on the right of figure \ref{fig:casestudy:boston}), each with shared scales along the horizontal axis which denotes observations over time.
%The case studies in section \ref{sec:casestudy} demonstrate the usefulness of the context provided by these visualizations.
%To create the visualization, we use Vega \cite{vega}, a declarative JSON-based visualization grammar.
%To provide the evaluation plots, we need the evaluation values and the observations that they occurred at.
%We log these values during the simulation our application runs, and then output the grammar in a tabular JSON-format that contains a row for each grammar element and it's associated evaluations.
%This tabular data is used by our Vega specification to produce the visualizations.
\begin{comment}
The app proceeds in multiple stages,
\begin{enumerate}
    \item First, a user selects a dataset of interest. We built support for two datasets, but our framework is generic enough for any arbitrary csv dataset.
    \item Following this choice, the input variables and output variable for a machine learning model must be specified.
    \item A machine learing model is then selected from a dropdown. We provide support for three models. However, this is for demonstration purposes only - the specification is agnostic to the choice of a machine learning model.
    \item Finally , a specification is input by the user of the app. On the press of a button, the model is trained and then evaluated on the selected dataset. The output monitored by the spec is passed off to the Vega module for further analysis.
\end{enumerate}
\end{comment}
\section{Evaluation}
\label{sec:casestudy}
In this section, we evaluate \AVOIRmethodname{}.variants through three real-world case studies.
Direct comparisons with existing work are impossible since no other work (to our knowledge) facilitates a general-purpose inference engine for online fairness auditing using arbitrary measures.
We can, however, implement VF's~\cite{bastani2019probabilistic} inference rules within \AVOIRmethodname{} (denoted as \AVOIRmethodname{}-VF).
Note that \AVOIRmethodname{}-VF sidesteps the assumptions of having a known data-generating distribution (made possible by \AVOIRmethodname{}'s reliance on confidence sets), making this variation a more practical and efficient algorithm. 
We denote \AVOIRmethodname{}-OB as the implementation that utilizes the abovementioned optimizations. 
Across the studies, the role of chosen threshold probabilities is similar to that of p-values in statistics.
Typical p-values tend to be $0.05$ and $0.1$, which we demonstrate in the RateMyProfs and COMPAS risk assessment study. 
In our case study of prior work~\cite{angwin2016machine}, we stick to the available definitions and thresholds used in the original analysis.
We expect that regulators will set the threshold probabilities on a case-by-case basis, e.g,, $0.15$ for illustration purposes in the adult income study.%, and we provide the adult income study with  $\Delta =0.15$ as an example.

%An important case study on the COMPAS dataset can be found in Appendix~\ref{sec:appendix:additional-case-studies}. 
\subsection{Rate My Profs}
\label{sec:casestudy:rmp}
\begin{figure}
    \includegraphics[width=0.5\linewidth]{avoir/images/ratemyprofs.png}
    \caption{Bounds for first half of a gender-fairness specification generated by \AVOIRmethodname{}-OB and \AVOIRmethodname{}-VF for \textit{RateMyProfs}, a real-world dataset. Vertical lines show the step at which the methods can provide a guarantee of failure for the upper bounds with $\Delta <= 0.05$. Blue horizontal line represents the constant term in the inequality.}
    \label{fig:casestudy:rmp}
\end{figure}
This section provides a detailed black-box machine learning model-based case study on a real-world dataset.
This case study uses the Rate My Professors (RMP) dataset~\cite{keymanesh2021fairness}. 
This dataset includes professor names and reviews for them written by students in their classes, ratings, and certain self-reported attributes of the reviewer.
Ratings are provided on a five-point scale (1-5 stars).
We use the preprocessing described in~\cite{keymanesh2021fairness} to infer the gender attribute for the professors.
This dataset is divided into an 80-20 split (train-test).
We then train a BERT-based transformer model~\cite{devlin2019bert} on the training split.
We use the implementation from the simpletransformers\footnote{https://simpletransformers.ai/} package.
The loss function chosen is the mean-squared error from the true ratings.
On the test set, we track a gender-fairness specification in the model outputs:
\begin{lstlisting}[columns=flexible, language=Python, basicstyle=\small]
(E[r > 3 | gender = F] / E[r > 3 | gender = M < 1.2) & 
(E[r > 3 | gender = M)] / E[r > 3 | gender = F] > 0.8)
\end{lstlisting}
We set the failure probability $\Delta = 0.05$. 
\texttt{OPT} is run after each batch (5 items/batch).
Figure~\ref{fig:casestudy:rmp} shows that \AVOIRmethodname{}-OB\footnote{OB = Optimized Bounds} can provide a guarantee in $\mathbf{2.5\%}$ fewer iterations than \AVOIRmethodname{}-VF. 
Note also that the OB guarantee provided tries to optimize for the failure probability while staying under the required threshold, remaining closer to the required threshold in subsequent steps.

\subsection{Adult Income}
\label{sec:casestudy:adult}
In this case study, we analyze the Adult income dataset~\citep{kohavi1996scaling}.
The historical dataset labels individuals from the 1994 census as having a \emph{high-income} ($>50$k a year) or not ($\leq50$k a year).
We consider this column of data as a black-box measurement. 
US Federal laws mandate against race and sex-based discrimination.
Thus, the specification we start our analysis with is a group fairness property for federal employees that monitors the difference of the proportions of individuals with sex marked male vs. female with a high income should be less than $0.5$.
In addition, we ensure that the difference between individuals with race marked white and non-white should have a difference of less than $0.5$.  
Thus, we use an \textit{intersectional} fairness criterion.
The associated specification is given below, where \texttt{h} is an indicator for whether an individual is \emph{high-income} is the binary classification output of our model:

\begin{lstlisting}[columns=flexible, language=Python, basicstyle=\small]
   (E[h | sex=M] - E[h | sex=F] < 0.5) & \ 
   (E[h | race=W] - E[h | race!=W] < 0.5)
\end{lstlisting}

In this example, we set the failure threshold probability $\Delta = 0.15$
\begin{figure}
    \centering
    \begin{subfigure}{0.48\linewidth}
    \includegraphics[width=\linewidth]{avoir/images/adult-left-initial.png}
    \caption{Group fairness for sex. Difference in ratio of high income (left subexpression).}
    \label{fig:casestudy:adult:specplot:left}
    \end{subfigure}
    %
    \begin{subfigure}{0.48\linewidth}
    \centering
    \includegraphics[width=\linewidth]{avoir/images/adult-right-initial.png}
    \caption{Group fairness for race. Difference in ratio of high-income earners (right subexpression). }
    \label{fig:casestudy:adult:specplot:right}
    \end{subfigure}
    \caption{\figtop{} Red dotted lines, the upper bounds of the value cannot be guaranteed to be under the threshold at the specified failure probability. \figbottom{} Guarantee possible with given data. Green lines represent the constant term, and dark blue is the empirical mean.}
\end{figure}
When run with this specification, the generated bounds cannot be achieved with the available data. 
We can then use the iterative refinement associated with subexpressions to analyze different components of the specification. 
The plot corresponding to the left subexpression is shown in Figure~\ref{fig:casestudy:adult:specplot:left} shows that guarantees cannot converge under the threshold with the given number of data samples. 
An auditor can now choose to either reduce the guarantee (i.e. increase $\Delta$) or increase the threshold. 
Next, analyzing the right subexpression, the race group fairness term can be guaranteed to be under the threshold (Figure~\ref{fig:casestudy:adult:specplot:right}).
Using this information, an auditor can make a decision to increase the threshold on the group fairness term for sex. 
As a hypothetical, suppose they increase it from $0.5$ to $0.55$ and rerun the analysis.
OB can provide a guarantee at this threshold within 870 steps, whereas VF can provide it at 960 steps, demonstrating a relative improvement of about $\mathbf{10.35\%}$.
Additionally, the optimal $\Delta$ split across the terms is $\approx (0.135, 0.36 * 10^4)$, which is far from the equal split allocated by VF.
The reason for this split is that increasing the threshold for the first time provides the optimizer with additional legroom to better distribute the failure probabilities between the two terms.

\subsection{COMPAS Risk Assessment}

The Correctional Offender Management Profiling for
Alternative Sanctions (COMPAS) recidivism risk score data is a popular dataset for assessing machine bias of commercial tools used to assess a criminal defendant's likelihood to re-offend.
The data is based on recidivism (re-offending) scores derived from software released by Northpointe and widely used across the United States for making sentencing decisions.
In 2016, \citet{angwin2016machine} at ProPublica released an article and associated analysis code critiquing machine bias associated with race present in the COMPAS risk scores for a set of arrested individuals in Broward County, Florida, over two years.
The analysis concluded that there were significant differences in the risk assessments of African-American and Caucasian individuals.
Northpointe pushed back in a report~\citep{dieterich2016compas} firmly rejecting the claims made by the ProPublica article; instead, they claimed that \citet{angwin2016machine} made several statistical and technical errors in the report.
In this case study, we use \AVOIRmethodname{} to study the claims of the two reports mentioned above. 
%First, we start with the data released by ProPublica and load it into a pandas-simulated DB.
We create a materialized view of the ProPublica dataset by reproducing the preprocessing steps in the publicly available ProPublica analysis  notebook\footnote{https://github.com/propublica/compas-analysis}.
We look at ``Sample A''~\citep{dieterich2016compas}, where the analysis of the ``not low'' risk assessments using a logistic regression model reveals a high coefficient associated with the factor associated with race being African-American.
In terms of a fairness metric, this corresponds to false positive rate (FPR) balance (predictive equality)~\citep{verma2018fairness} metrics. 
The associated specification in \AVOIRmethodname{} grammar would be

\begin{lstlisting}[columns=flexible, language=Python, basicstyle=\small]
   E[hrisk | race=African-American & recid=0] / 
   E[hrisk | race=Caucasian & recid=0] < 1.1
\end{lstlisting}

Where \verb|hrisk| is an indicator for high-risk assessments made by the \emph{black-box} COMPAS tool as defined by ~\citet{angwin2016machine},  \verb|recid| is an indicator for re-offending within two years of first arrest, and a $90\%$-rule is used as the threshold. 
We choose a failure threshold probability of $\Delta = 0.1$, with the optimization run after every batch of $5$ samples.
\AVOIRmethodname{} finds that when the decisions are made sequentially, online, the assertion for specification violation cannot be made with the required failure guarantee.

By analyzing the component subexpressions, one can glean that \AVOIRmethodname{} cannot optimize since the lower FPR in the denominator (FPR for Caucasian individuals) increases the overall variance and limits the ability to optimize for guarantees. 
We follow this analysis by using the reciprocal specification, i.e.,
\begin{lstlisting}[columns=flexible, language=Python, basicstyle=\small]
   E[hrisk | race=Caucasian & recid=0] /
   E[hrisk | race=African-American & recid=0] > 0.9
\end{lstlisting}

We find that the specification is guaranteed to be violated with a confidence of over $1 - \Delta = 0.9$ probability, and \AVOIRmethodname{} can detect this violation within about half the number of available assessments (3350 steps) when run in an online setting.
Figure~\ref{fig:casestudy:compas:propublica} demonstrates the progression of the tracked expectation term. 
Thus, if deployed with the corrected specification, \AVOIRmethodname{} would be able to alert Northpointe/an auditor of a violation/potentially-biased decision-making tool.

\begin{figure}
    \begin{subfigure}{0.48\linewidth}
        \centering
        \includegraphics[width=\linewidth]{avoir/images/compas-propublica-et.png}
        \caption{(ProPublica) COMPAS, ``Sample A'' False Positive Rate Bias specification required to \emph{above} the $10\% \implies 0.9$ threshold converges to a value that can be guaranteed to be \emph{under} the required threshold.}
        \label{fig:casestudy:compas:propublica}
    \end{subfigure}
    \hfill
    \begin{subfigure}{0.48\linewidth}
        \centering
        \includegraphics[width=\linewidth]{avoir/images/compas-northpointe-et.png}
        \caption{(Northpointe) ``Sample B'' analysis done by Northpointe using False Discovery Rate that opposed the ProPublica reports.}
        \label{fig:casestudy:compas:northpointe}
    \end{subfigure}
    \caption{COMPAS dataset case study.}
\end{figure}


The Northpointe report~\citep{dieterich2016compas} makes several claims about the shortcomings of this analysis.
One of the primary claims is that \citet{angwin2016machine} used an analysis based on ``Model Errors'' rather than ``Target Population Errors''.
In fairness specification terms, this refers to the difference between a False Positive Rate (FPR) balance vs. False Discovery Rate (FDR) balance, i.e., balancing for predictive parity over predictive equality. 
In probabilistic terms, the difference amounts to comparing $\Pr[\hat{Y} = 1 | Y = 0, g=1, 2]$ (FPR) vs $\Pr[Y = 0 | \hat{Y} = 1, g=1, 2]$ (FDR), where $\hat{Y}$ refers to the decision made by the algorithm, $Y$ refers to the true value, and $g = 1, 2$ reflects group membership~\citep{verma2018fairness}.
This analysis is run on the dataset subset dubbed ``Sample B''.
To test their hypothesis, we reproduce the corresponding preprocessing steps and run both versions (numerator and denominator being Caucasian) of the corresponding specification under the same setup as earlier. 
Despite the point estimate being within the required threshold, we find that neither version can be guaranteed with the required confidence in the given data.
Due to the paucity of space, we describe only one of the two variants with the corresponding figure (Figure~\ref{fig:casestudy:compas:northpointe}).
\begin{lstlisting}[columns=flexible, language=Python]
   E[recid=0 | race=Caucasian & hrisk] /
   E[recid=0 | race=African-American & hrisk] > 0.9
\end{lstlisting}


We note that the Northpointe report~\citep{dieterich2016compas} does not provide confidence intervals for their claim. 
Further, even though the report does not release associated code, the point estimates of the False Discovery Rates (FDRs) match those present in the report, which increases our confidence in our \AVOIRmethodname{}-based analysis. 

The back-and-forth exchange has been the subject of much discussion in academic and journalistic publications~\citep{feller2016computer, washington2018argue}.
Seminal work by \citet{kleinberg2017inherent} proved the impossibility of simultaneously guaranteeing certain combinations of fairness metrics.
While \AVOIRmethodname{} cannot circumvent this problem, its usage can help audit claimed guarantees on defined metrics.
We conclude this case study by noting that \AVOIRmethodname{} lends itself to successful analysis that is not possible with the VF implementation available online, which only provides support for a predefined set of specifications and requires access to a data-generating function.
In addition, we choose $0.1$ as the failure probability because it is one of the thresholds used in \cite{angwin2016machine}.  
We set it to the highest used threshold to allow leeway for the claim by Northpointe.
Even under this lax threshold, the analysis by Northpointe fails.

\section{Discussion} %\& Future Work}
The case studies presented in the previous section demonstrate the ability of our tools to provide vital context when deciding how to refine a model or fairness specification.
Although this contextual information makes decisions easier, it is not always clear how one should alter a specification in light of a violation and its relevant context.
%For example, in case study B, the decision was made that the threshold could be lowered from 2. However, it is not obvious how much we should lower this threshold. Further complicating the matter is the fact that we have options when changing the specification. We can lower the threshold constant (2), or we can add a constant multiplier to the expectation of females with high income.
To assist in these decisions, we are currently examining ways work to suggest edits that are likely to achieve the desired intent of a developer.
Using our visual analysis tool for refinement, we can gather edits from developers and then use that data to learn iterative changes to the syntax tree of the specification.
In addition to improving the usability of our tools for making fairness specification refinements, we also envision a more scalable framework.
Our case studies look at a single model with respect to a single dataset. However, real-world deployment of machine learning often contain many clients with models that may differ.
However, real-world deployment of machine learning often contain many clients with models and datasets that may evolve and drift over time.
We take it as future work to study the efficient monitoring of machine learning behavior with respect to a fairness specification in a distributed context, enabling horizontal scalability.
We believe techniques such as decoupling the observation of data and the reporting results from the monitoring of the results are promising and can lead to the desired scalability.

%\vspace{-0.2in}
\section{Conclusion}

We presented the \AVOIRmethodname{} framework to easily define and monitor fairness specifications online and aid in the refinement of specifications. \AVOIRmethodname{} is easy to integrate within modern database systems but can also serve as a standalone system evaluating whether black-box machine learning models meet specific fairness criteria on specific datasets (including both structured and unstructured data) as described in our case studies.
\AVOIRmethodname{} extends the grammar from Fairness Aware Programming~\cite{albarghouthi2019fairness} with operations that enhance expressiveness.
In addition, we derive probabilistic guarantees that improve the confidence with which specification violations are reported.
Through case studies, we demonstrate that \AVOIRmethodname{} can provide users with insights and context that contribute directly to refinement decisions.
To understand the robustness of \AVOIRmethodname{}, we evaluated it along two dimensions: the data/ML model used and changing parameters (thresholds, fairness definitions).
We demonstrated the robustness of the data/model used by evaluating three datasets of varying domains and types (criminal justice - COMPAS, text classification - RateMyProfs, census data - Adult Income). 
For robustness to the thresholds, we used varying failure probability levels ($0.05$, $0.1$, $0.15$) in our case studies.
Note that any probability thresholds over these values for the corresponding studies would converge in fewer iterations, while lower thresholds would require additional data samples.
Our framework builds the foundation for further improvements in fairness specification, auditing, and verification workflows.
Although contextual information from \AVOIRmethodname{} makes decisions more straightforward, it is not always clear how to alter a specification in light of a violation and its relevant context.

To assist in these decisions, we are currently examining mechanisms that suggest edits that are likely to achieve the desired intent of a model developer.
We plan to extend this work to provide intelligent specification refinement suggestions and support distributed machine learning settings.
In addition to improving the usability of our tools for making fairness specification refinements, we also envision a more scalable framework.
Our case studies looked at a single model with respect to a single dataset. 
However, real-world deployment of machine learning often contains many clients with models and datasets that may evolve and drift over time.
We also expect to examine efficient monitoring of machine learning behavior for a fairness specification in a distributed context, enabling horizontal scalability.
We believe techniques such as decoupling the observation of data and reporting results from monitoring the results are promising and can lead to the desired scalability.
%We also plan to explore the use of \AVOIRmethodname{} for fair ranking problems and tailored database integration.

\newpage \section{Reproducibility}
\label{sec:reproducibility}
To enhance the reproducibility of our work, on the theoretical side, all proofs (with the necessary assumptions) are provided in the appendix. 
Specifically, proofs for the inference engine are in Appendix~\ref{sec:appendix:inference-rules}, and proofs for the correctness of bounds are provided in Appendix~\ref{sec:appendix:confseq}. 
Theorem~\ref{theorem:better-stopping}, which shows how \AVOIRmethodname{} improves over prior work is proved in Appendix~\ref{sec:appendix:optimality}.
To reproduce the results of the case studies in the paper, each case study is encapsulated inside a Jupyter notebook. 
These notebooks are attached along with the source code for \AVOIRmethodname{}.
In addition, all datasets used for generating results for the case studies are also attached in the submitted supplementary documentation.
Finally, the model weights used for the \textit{RateMyProfs} study for exact reproduction are provided in a dropbox folder hosted at  \url{https://www.dropbox.com/sh/n5o4vswnkxv34zr/AABthgLMaYL3MuA0KC39Z1G8a?dl=0}.
\begin{comment}
It is important that the work published in ICLR is reproducible. Authors are strongly encouraged to include a paragraph-long Reproducibility Statement at the end of the main text (before references) to discuss the efforts that have been made to ensure reproducibility. This paragraph should not itself describe details needed for reproducing the results, but rather reference the parts of the main paper, appendix, and supplemental materials that will help with reproducibility. For example, for novel models or algorithms, a link to a anonymous downloadable source code can be submitted as supplementary materials; for theoretical results, clear explanations of any assumptions and a complete proof of the claims can be included in the appendix; for any datasets used in the experiments, a complete description of the data processing steps can be provided in the supplementary materials. Each of the above are examples of things that can be referenced in the reproducibility statement. This optional reproducibility statement will not count toward the page limit, but should not be more than 1 page.
\end{comment}
\section{Inference Rules}
\label{sec:appendix:inference-rules}
In Figure~\ref{fig:inference}, we provide a set of rules that can be used to determining the constraints and guarantees associated with a specification.
We represent
\[ 
X \odot Y : (E, \epsilon, \delta) \equiv \Pr\left(| \E[X] \odot \E[Y] - E | \geq \epsilon \right) \leq \delta
\]
where $\odot$ represents a binary operator.
Constraints are represented in curly brackets $\{ \}$.
%\pmcomment{Need to explain that constraints carry over in missing cases in Figure!\ref{fig:inference}}

%\pmcomment{Add proofs here for inference rules.}
The proof of correctness for each inference rule starts from the assumptions above the horizontal line and derives the assertions below. 
These proofs ise ideas similar to those in \cite{bastani2019probabilistic}.
We reproduce the proofs in Appendix~\ref{sec:appendix:constraint:proofs} here for completeness.
%We provide them here for completeness.
Note that the assertions in the base case (elementary subexpressions) can be arrived at by applying the Adaptive Hoeffding INequality, ($\rm{AIN}$). 

\begin{figure}[!htbp]
%\begin{minipage}{\textwidth}
\centering
\[
\dfrac{X: \left(\BarE[X], \epsilon_X, \delta_X\right), Y: \left(\BarE[Y], \epsilon_Y, \delta_Y\right)}{X \pm Y: \left(\BarE[X] \pm \BarE[Y], \epsilon_X + \epsilon_Y, \delta_X + \delta_Y\right)}
\]
\[
\dfrac{X: \left(\BarE[X], \epsilon_X, \delta_X\right), Y: \left(\BarE[Y], \epsilon_Y, \delta_Y\right)}{ X \times Y: (\BarE[X] \BarE[Y], \epsilon_X \epsilon_Y + \BarE[X] \epsilon_Y + \BarE[Y] \epsilon_X, \delta_X + \delta_Y)} 
\]
\[
\dfrac{X: \left(\BarE, \epsilon, \delta \right), \BarE - \epsilon > 0}{ X^{-1}: \left(\BarE^{-1}, \frac{\epsilon}{\BarE (\BarE - \epsilon)}  , \delta \right)}\text{ (Inverse)} \hspace{0.2in} \dfrac{X: \left(\BarE, \epsilon, \delta \right)}{ X^{-1}: \left(\BarE^{-1}, \frac{\epsilon}{\BarE (\BarE - \epsilon)}  , \delta \right), \{ \BarE - \epsilon > 0\}} \text{ (Inverse Constr.)}
\]
\[
\dfrac{X: \left(\BarE, \epsilon, \delta\right), \BarE - \epsilon > c}{\psi \equiv X > c: (T, \delta)} \text{ (True)} \hspace{0.2in} \dfrac{X: \left(\BarE, \epsilon, \delta \right), \BarE  + \epsilon < c}{\psi \equiv X < c: (F, \delta)} \text{ (False)} 
\] 
\[
\dfrac{X: \left(\BarE, \epsilon, \delta\right)}{\psi \equiv X > c: (T, \delta), \{\BarE - \epsilon > c\}} \text{ (True Constr.)} \]
\[
\dfrac{X: \left(\BarE, \epsilon, \delta \right)}{\psi \equiv X < c: (T, \delta), \{\BarE + \epsilon < c\}} \text{ (False Constr.)} 
\] 
\[
\dfrac{\psi_1: (\B_1, \delta_1), \psi_2: (\B_2, \delta_2)}{\psi_1 \wedge \psi_2: (\B_1 \wedge \B_2, \delta_1 + \delta_2)} \text{ (and)} \hspace{0.2in} \dfrac{\psi_1: (\B_1, \delta_1), \psi_2: (\B_2, \delta_2)}{\psi_1 \vee \psi_2: (\B_1 \vee \B_2, \delta_1 + \delta_2)} \text{ (or)}
\]
\[
\dfrac{\psi_1: (\B_1, \delta_1), \{C_{11, \dots, 1k}\}, \psi_2: (\B_2, \delta_2), \{C_{21, \dots, 2m}\}}{\psi_1 \wedge \psi_2: (\B_1 \wedge \B_2, \delta_1 + \delta_2), \{C_{11, \dots, 1k}, C_{21, \dots, 2m}\}} \text{ (and constr.)} \]
\[
\hspace{0.2in} \dfrac{\psi_1: (\B_1, \delta_1), \{C_{11, \dots, 1k}\}, \psi_2: (\B_2, \delta_2)}{\psi_1 \vee \psi_2: (\B_1 \vee \B_2, \delta_1 + \delta_2),  \{C_{11, \dots, 1k}\} \vee \{C_{21, \dots, 2m}\}} \text{ (or constr.)}
\]
\caption{Inference rules used to guarantees for expressions.The inference rules for each compound expression build on the union bound, triangle inequality, and structural induction approach described by \cite{bastani2019probabilistic}.}
\label{fig:inference}
%\end{minipage}
\end{figure}
\subsection{Inference rules with Constraints}
\label{sec:appendix:constraint:proofs}

%We define guarantees for concentration using an appropriate concentration inequality. 
%Examples are provided in the Appendix~\ref{sec:appendix:inequality}  
%\paragraph{Correctness} 
In Section~\ref{sec:theoretical:propagation} we provided the proofs for $X\pm Y$, $X > c$.
In the following text we provide the proofs for the remainder of the inference rules. 

\paragraph{Product} Starting with $\phi_X = X: (\BarE[X], \epsilon_X, \delta_X)$, $\phi_Y = Y: (\BarE[Y], \epsilon_Y, \delta_Y)$. 
First, from union bound, both of these hold true with probability at least $1 - \delta_X - \delta_Y$.
Then,
\begin{align*}
|\E[X]| &= |\BarE[X] - \BarE[X] + \E[X]|\\
        &\leq ||\BarE[X]| +  |\BarE[X] + \E[X]| \\
        &\leq ||\BarE[X]| + \epsilon_X \\
\end{align*}
We have
\begin{align*}
    |\BarE[X]\BarE[Y] - \E[XY]| &=  |\BarE[X]\BarE[Y]  - \E[X]\E[Y]| & \text{(as $X, Y$ bernoulli)}\\
    &= |\BarE[X]\BarE[Y] - \BarE[X]\E[Y] + \BarE[X]\E[Y]  - \E[X]\E[Y]| \\
    &= |\BarE[X](\BarE[Y] - \E[Y]) + \E[Y] (\BarE[X]\  - \E[X])| \\
    &\leq |\BarE[X]||(\BarE[Y] - \E[Y])| + |\E[Y]||(\BarE[X]\  - \E[X])| \\
    &\leq |\BarE[X]| \eps_Y + |\E[Y]| \eps_X \\
    &\leq |\BarE[X]| \eps_Y + (|\BarE[Y]| + \eps_Y) \eps_X \\
    &= |\BarE[X]| \eps_Y + |\BarE[Y]| \eps_X + \eps_X \eps_Y
\end{align*}
Therefore, $X \times Y: (\BarE[X] \BarE[Y], \epsilon_X \epsilon_Y + \BarE[X] \epsilon_Y + \BarE[Y] \epsilon_X, \delta_X + \delta_Y)$

\paragraph{Inverse/Inverse constr.} Assume $X: \left(\BarE, \epsilon, \delta \right)$ and $\BarE - \epsilon > 0$.
Instead, in the constrained case, we start with only the prior assumption i.e., $X: \left(\BarE, \epsilon, \delta \right)$
Then,
\begin{align*}
    |\E[X]| &= |\E[X] - \BarE[X] + \BarE[X]| \\
    &\leq |\E[X] - \BarE[X]| + |\BarE[X]| \\
    &\leq \epsilon_X + |\BarE[X]|
\end{align*}
i.e., $|\E[X]| \leq \epsilon_X + |\BarE[X]|$. 
Also,
\begin{align*}
    |\E[X]^{-1} - \BarE[X]^{-1}| &= \left|\frac{\BarE[X]^{-1} - \E[X]^{-1}}{\BarE[X]  \E[X]^{-1}}\right| \\
    &\leq \frac{\epsilon}{|\E[X]|  |\BarE[X]|} \\
    &\leq \frac{\epsilon}{|\E[X]| (\E[X] - \epsilon_X)}
\end{align*}

where the last step follows from the previous derivation and if $\E[X] - \epsilon_X > 0$.
The latter condition enforces that the sign of the inequality does not change.
VF adds this as a precondition; we add it as a post-constraint.

\paragraph{Boolean Operators}
Starting from $\psi_1: (b_1, \delta_1)$, $\psi_2: (b_2, \delta_2)$, we can apply the union bound for $\psi_1 \wedge \psi_2$, $\psi_1 \vee \psi_2$ to derive the rules for and/or.
%\pmcomment{TODO: Complete Proofs}
Similarly, constraints follow the semantics specified by the rules as they also follow from the union bound.




\subsection{Inferred Optimization Problem}
\label{sec:appendix:inferrence-rules:opt}
For a given overall specification $\psi$, suppose $(\epsilon_i, \delta_i$), $i \in \{1, \dots, n\}$ represents the concentration bounds associated with each constituent elementary subexpression. 
Using the aforementioned inference rules, we can derive the overall $\delta_T = \sum\limits_{i}\delta_i$, along with a set of (say) $K$ constraints 
\[
g_k(\epsilon_{1}, \dots, \epsilon_{n}, \BarE[X_1], \dots, \BarE[X_n]) \leq \epsilon_k
\]
where 
\[
\epsilon_k = \left|c_k - \BarE[f(\BarE[X_1], \dots, \BarE[X_n])]\right|
\]
denotes the maximum allowed margin for the $\text{k}^{\text{th}}$ inequality subexpression (i.e. having form \texttt{<ETerm> <comp-op> c}).
The objective is to minimize the overall failure probability $\delta_T$.
The overall optimization problem can then be formulated as shown in \ref{eq:optimization},
having $n$ optimizaiton variables $\delta_i$ and $2n + K$ constraints (bounds on $\delta_i$ provide the $2n$ constraints).  
A developer using \AVOIRmethodname{} inputs a required acceptable upper bound of failure probability $\Delta$.
If the solution to the optimization problem $\delta_T^* = \sum_i \delta_i \leq \Delta$, then the optimization can conclude with the required confidence in the proved guarantee.
At this point, the developer may choose to terminate \AVOIRmethodname{}.
However, using Corollary~\ref{thm:adaptive-stopping:anytime}, they may continue to run and refine the estimates.

\section{Concentration bounds}
The adaptive Hoeffding inequality~\citep{zhao2016adaptive, bastani2019probabilistic}. 

%\newtheorem{theorem}{Theorem} % TODO: move to my_definitions.tex

\begin{theorem}
\label{thm:adaptive-stopping}
%\pmcomment{No dependence on $Var[X]$}
Given a Bernoulli random variable X with distribution $P_X$. Let $\{X_i \sim P_X\}, i \in \N$ be i.i.d samples of $X$. Let 
\[
\BarE_t[X] = \frac{1}{t}\sum\limits_{i=1}^{t}X_i.
\]
Let $\gT$ be a random variable on $\N \cup \{\infty\}$ such that $\Pr[\gT < \infty] = 1$, and let
\[
    \epsilon(\delta, t) = \sqrt{ \frac{ \frac{3}{5}\log{(\log_{1.1}{t} + 1)} + \frac{5}{9}\log{(24/\delta)} } {t}}
\]
Then, for any $\delta \in \R_+$, we have
\[
  \Pr[|\BarE_\gT[X] - \E[X]| \leq \epsilon(\delta, \gT)|] \geq 1 - \delta 
\]
.
\end{theorem}



Theorem~\ref{thm:adaptive-stopping} provides a mechanism for choosing the stopping time using arbitrary methods for a fixed $\delta$. 
Note that in general, any adaptive concentration inequality suffices; we use the Hoeffding inequality that does not depend on the empirical variance but is frequently used in scenarios dealing with bounded rvs.
However, we use confidence intervals to visualize the evolution of sub-expressions (and overall specification) over the sequence of observations. 
For doing so, we require an additional result

\begin{theorem}\cite[Proposition 1, Lemma 1]{zhao2016adaptive}
Let $S_n = \sum_{i=1}^n X_i$ be a random walk from i.i.d. random variables $X_1, \dots, X_t \sim D$. For any $\delta > 0$,
$$\Pr[S_\gT \geq f(\gT)] \leq \delta$$
for any stopping time $\gT$ if and only if
$$\Pr\left[\exists n, S_t \geq f(t) \right] \leq \delta$$

\label{thm:zhao:adaptive-hoeffding:anytime}
\end{theorem}

\begin{corollary}
\label{thm:adaptive-stopping:anytime}
For any $\delta > 0$, 
\[
  \Pr[|\BarE_\gT[X] - \E[X]| \leq \epsilon(\delta, \gT)|] \geq 1 - \delta 
\]
for any stopping time $\gT$ if and only if
\[
\Pr\left[\forall t, |\BarE_t[X] - \E[X]| \leq \epsilon(\delta, t)| \right] \geq 1 - \delta
\]
\end{corollary}
\begin{proof}
%\pmcomment{need to correct this} 
Follows directly from applying Theorem~\ref{thm:zhao:adaptive-hoeffding:anytime} to Theorem~\ref{thm:adaptive-stopping}.
\end{proof}

Intuitively, Theorem~\ref{thm:adaptive-stopping} holds since we can choose an adversarial stopping rule for $\gT$ that terminates as soon as the boundary for $\epsilon(\delta, t)$ is crossed~\citep{zhao2016adaptive}. 
Thus, when we establish a bound with a stopping rule, as long as the underlying distribution remains unchanged, the bound will hold prior to and after the stopping rule is enforced.
Theorem~\ref{thm:adaptive-stopping:anytime} implies that once we choose an optimal bound for each subexpression, we can extend the bounds derived using Theorem~\ref{thm:adaptive-stopping} to following observations with the guarantees for the subexpressions still holding true.
%Note that this does not necessarily imply that the specification will still be True/False with a bounded failure probability since the truth value of the specification depends on the empirical mean.



\section{Confidence Sequences}
\label{sec:appendix:confseq}
In this section, we show that the estimates generated from AVOIR form a confidence set (Theorem~\ref{thm:conf-seq}).
First, we assume the existence of a concentration sequence for the mean of each elementary subexpression (eg., Theorem~\ref{thm:adaptive-stopping} can provide one). 
That is, we need a function $\epsilon(t, \delta)$ such that
\begin{equation}
    \Pr[\forall t \geq 1, |\BarE_t[X] - \E[X]| \leq \epsilon(t, \delta_X)] \geq 1 - \delta_X.
    \label{eqn:conf-seq:elementary:assumed}
\end{equation}
For convenience of exposition, we denote such adaptive inequality functions as $\rm{AIN}$.
For any $\rm{AIN}$ to be usable with \AVOIRmethodname{}, we require $\eps(t, \delta)$ to be monotonically non-increasing in $\delta$ and $n$. 
We expect this to be the case for most $\rm{AIN}$, since increasing the number of observations of increasing the failure threshold should allow for additional concentration around the mean.
For example, the adaptive Hoeffding inequality (Theorem ~\ref{thm:adaptive-stopping}) follows this assumption.
%\pmcomment{Should we explain that this should be natural? Since more values/more failure threshold would allow for more concentration.
%SP: Yes - makes sense.}
\begin{comment}
\begin{lemma}
Given two confidence sequences constructed using $\rm{AIN}$ for an elementary subexpression, with thresholds $\delta_1, \delta_2$, at any time step t,
\begin{equation*}
    \delta_1 \leq \delta_2 \implies \Pr[|\BarE_t[X] - \E[X]| > \epsilon(t, \delta_2)] \geq \Pr[|\BarE_t[X] - \E[X]| > \epsilon(t, \delta_1)]
\end{equation*}
\label{lemma:conf-seq:delta-ineq}
\end{lemma}
\begin{proof}
$\rm{AIN}$ is monotonically non-increasing in $\delta$, thus
\begin{align*}
    \eps(n, \delta_1) &\geq \eps(n, \delta_2) \\
    &\implies \left(\BarE_t[X] - \eps(t, \delta_2), \BarE_t[X] + \eps(t, \delta_2)\right) \subseteq \left(\BarE_t[X] - \eps(t, \delta_1),\BarE_t[X] + \eps(t, \delta_1)\right)\\
    &\implies \Pr[|\BarE_t[X] - \E[X]| > \epsilon(t, \delta_2)] \geq \Pr[|\BarE_t[X] - \E[X]| > \epsilon(t, \delta_1)]
\end{align*}
%$CI_1 \subseteq CI_2$ \pmcomment{TODO}
\end{proof}
\end{comment}
Second, we assume that, except in degenerate cases, \AVOIRmethodname{} terminates (see corollary~\ref{corollary:termination} for termination criteria). 
We now prove Theorem~\ref{thm:conf-seq}.
\begin{proof}
First, we will prove that the estimates for \textit{elementary} subexpressions are a confidence sequence.
Following this, using the inference rules from Appendix~\ref{sec:appendix:inference-rules}, we will show that the estimates for every compound expression are also a confidence sequance.
\paragraph{Elementary subexpressions} Consider a specification $\psi$ consisting of \textit{elementary} subexpressions $X_1, \dots, X_n$.
At stopping time $\gT$, let
\begin{equation}
    \phi^\gT_{X_i} \eqdef X_i: (\BarE_\gT[X_i], \epsilon(\gT, \delta_{X_i}), \delta_{X_i})
    \label{eqn:conf-seq:elementary:stopping}
\end{equation}
be the stopping time estimates. 
Then, from the termination criterion, a solution to the optimization problem \texttt{OPT} exists, i.e, 
\begin{equation}
    \Delta  \geq \sum_i \delta_{X_i}
    \label{thm:conf-seq:proof:delta-inequality}
\end{equation}

The sequence of bounds claimed by \AVOIRmethodname{} are
\begin{align}
    \epsilon_{X_i}(t) = 
    \begin{cases}
        \epsilon(\Delta, t), & t < \gT, \\
        \epsilon(\delta_{X_i}, t), & t \geq \gT
    \end{cases}
    \label{eqn:conf-seq:epsilon:def}
\end{align}

From \eqref{thm:conf-seq:proof:delta-inequality} and the optimization constraint $\delta_i \in [0, 1]$ we have $\Delta \geq \delta_{X_i}$. 
From the non-decreasing behavior of $\rm{AIN}$
%We can then apply Lemma~\ref{lemma:conf-seq:delta-ineq} to get 
\begin{equation}
    %\Pr[|\BarE_t[X_i] - \E[X_i]| > \epsilon(\Delta, t)| \geq \Pr[|\BarE_t[X_i] - \E[X_i]| > \epsilon(\delta_{X_i}, t)| 
    \eps(\Delta, t) \leq \eps(\delta_i, t)
    \label{eqn:conf-seq:lemma-derivation}
\end{equation}

Now
\begin{align*}
    \Pr[&\forall t \geq 1, |\BarE_t[X_i] - \E[X_i]| \leq \epsilon_{X_i}(t)] \\
                           &= 1 -  \Pr[\exists t \geq 1, |\BarE_t[X_i] - \E[X_i]| > \epsilon_{X_i}(t)] \\
                           &= 1 - \Pr\left[\bigcup\limits_{t \geq 1} \left\{|\BarE_t[X_i] - \E[X_i]| > \epsilon_{X_i}(t)\right\}\right] \\
                           &= 1 - \Pr\left[\bigcup\limits_{t = 1}^{\gT-1} \left\{|\BarE_t[X_i] - \E[X_i]| > \epsilon_{X_i}(t)\right\} \cup \bigcup\limits_{t \geq \gT} \left\{|\BarE_t[X_i] - \E[X_i]| > \epsilon_{X_i}(t)\right\} \right]\\
                           & \text{(associativity of $\cup$)} \\
                           &= 1 - \Pr\left[\bigcup\limits_{t = 1}^{\gT-1} \left\{|\BarE_t[X_i] - \E[X_i]| > \epsilon(\Delta, t)\right\} \cup \bigcup\limits_{t \geq \gT} \left\{|\BarE_t[X_i] - \E[X_i]| > \epsilon(\delta_{X_i}, t)\right\} \right] \\
                           & \text{(From~\ref{eqn:conf-seq:epsilon:def})}\\
                           &= 1 - \Pr\left[\bigcup\limits_{t = 1}^{\gT-1} \left\{|\BarE_t[X_i] - \E[X_i]| > \epsilon(\delta_{X_i}, t) \cup |\BarE_t[X_i] - \E[X_i]| \in (\epsilon(\Delta, t), \epsilon(\delta_{X_i}, t)] \right\} \cup \right. \\ &\left. \bigcup\limits_{t \geq \gT} \left\{|\BarE_t[X_i] - \E[X_i]| > \epsilon(\delta_{X_i}, t)\right\} \right] \text{ (Using \ref{eqn:conf-seq:lemma-derivation})} \\
                           &= 1 - \Pr\left[\bigcup\limits_{t = 1}^{\gT-1} \left\{ |\BarE_t[X_i] - \E[X_i]| \in (\epsilon(\Delta, t), \epsilon(\delta_{X_i}, t)] \right\} \cup  \bigcup\limits_{t \geq 1} \left\{|\BarE_t[X_i] - \E[X_i]| > \epsilon(\delta_{X_i}, t)\right\} \right]\\
                           &\text{(Rearranging)}\\
                           &\geq 1 - \Pr\left[\bigcup\limits_{t \geq 1} \left\{|\BarE_t[X_i] - \E[X_i]| > \epsilon(\delta_{X_i}, t)\right\} \right] \\
                           &= 1 - \Pr\left[ \exists t \geq 1, |\BarE_t[X_i] - \E[X_i]| > \epsilon(\delta_{X_i}, t) \right] \\
                           &\geq 1 - \delta_{X_i}
\end{align*}
\begin{comment}
\begin{align*}
    \Pr[&\forall t \geq 1, |\BarE_t[X_i] - \E[X_i]| \leq \epsilon_{X_i}(t)] \\
                           &= 1 -  \Pr[\exists t \geq 1, |\BarE_t[X_i] - \E[X_i]| > \epsilon_{X_i}(t)] \\
                           &= 1 - \Pr\left[\bigcup\limits_{t \geq 1} |\BarE_t[X_i] - \E[X_i]| > \epsilon_{X_i}(t)\right] \\
                           &\geq  1 - \sum\limits_{t \geq 1} \Pr[|\BarE_t[X_i] - \E[X_i]| > \epsilon_{X_i}(t)]  & \text{(union bound)}\\
                           &\geq 1 - \sum\limits_{t = 1}^{\gT-1} \Pr[|\BarE_t[X_i] - \E[X_i]| > \epsilon(\Delta, t)] - \sum\limits_{t \geq \gT} \Pr[|\BarE_t[X_i] - \E[X_i]| > \epsilon(\delta_{X_i}, t)]  & \text{(From~\ref{eqn:conf-seq:epsilon:def})}\\
                           &\geq 1 - \sum\limits_{t = 1}^{\gT-1} \Pr[|\BarE_t[X_i] - \E[X_i]| > \epsilon(\delta_{X_i}, t)] - \sum\limits_{t \geq \gT} \Pr[|\BarE_t[X_i] - \E[X_i]| > \epsilon(\delta_{X_i}, t)] & (\text{\eqref{eqn:conf-seq:lemma-derivation}}) \\
                           &\geq 1 - \sum\limits_{t \geq 1} \Pr[|\BarE_t[X_i] - \E[X_i]| > \epsilon(\delta_{X_i}, t)] \\
                           & \geq 1 - \delta_{X_i} 
\end{align*}
\end{comment}
where the last step follows from the definition of the adaptive concentration bound used.
Thus, $\epsilon_{X_i}(t)$ defines a $1 - \delta_{X_i}$ confidence sequence for $\E[X_i]$.
\paragraph{Compound subexpressions} 
Consider a non-specification compound \texttt{<ETerm>} $C_j$ consisting of \textit{elementary} subexpressions with indices $\etC_j = \{\{j_1, j_2, \dots, j_{M}\}\}$ as the decision r.v.s, i.e, $X_{j_1} \dots, X_{j_{M}}$.
Note that $\etC_j$ is a multiset as the same expression could occur multiple times within $C_j$. 
At stopping time $\gT$, 
\begin{equation}
    \phi_{C_j}^\gT: (\BarE_\gT[C_j], \delta_{C_j}, \eps_{C_j})
\end{equation}
where $\BarE_\gT[C_j], \delta_{C_j}, \eps_{C_j}$ are the corresponding values computed through the inference rules.
In general, we denote by 
\begin{equation}
\BarE_t[C_j], \delta_{C_j}(t), \eps_{C_j}(t) = \rm{INFER}(\phi^t_{X_{j_1}}, \dots, \phi^t_{X_{j_M}})
\label{eqn:conf-seq:compound:inf1}
\end{equation}
the values inferred at time step $t$, where $\rm{INFER}$ denotes the inference rules. 
Now,
\begin{align*}
    \Pr[&\exists t \geq 1, |\E[C_j] - \BarE[C_j]| > \eps_{C_j}(t)]\\
        &\leq \Pr\left[\bigcup\limits_{i = 1}^{M} \exists t \geq 1,  \neg \phi_{X_{j_i}}^t \right] & \text{(From \ref{eqn:conf-seq:compound:inf1})} \\
      &\leq \sum\limits_{i\in \etC_j}\Pr\left[\exists t \geq 1, \neg \phi^t_{X_{j_i}}\right] & \text{(union bound)} \\
      &= \sum\limits_{i\in \etC_j}\Pr\left[\exists t \geq 1, |\BarE_t[X_{j_i}] - \E_t[X_{j_i}| > \epsilon_{X_{j_i}}(t)\right] & \text{(definition of $\phi^t_{X_{j_i}}$ )} \\
      &\leq \sum\limits_{i\in \etC_j} \delta_{X_{j_i}} & \text{(elementary subexpressions)} \\
      &\leq \delta_{C_j} & \text{(applying \ref{eqn:conf-seq:compound:inf1} for $t = \gT$)}
\end{align*}
Therefore $\eps_{C_j}(t)$ defines a $1 - \delta_{C_j}$ confidence sequence for $\E[C_j]$

A similar proof can be constructed for any \texttt{<spec>}. 
Consider any specification $\psi_k$. Let
\begin{equation}
    \psi_k^t : (\hat{b}_{\psi_k}(t), \delta_{\psi_k}(t))
\end{equation}
where $\hat{b}_{\psi_k}(t) \subseteq \{T, F\}$ is the inferred value and $\delta_{\psi_k}(t)$ corresponds to the confidence for the assertion at time $t$. 
Let the \textit{elementary} subexpressions involved be $X_{k_1}, \dots, X_{k_D}$ corresponding to the index multiset $\etB_k = \{\{k_1, \dots, k_D \}\}$.
Denote $b_{\psi_k}$ as the true value of $\psi_k$, and $\delta_{\psi_k}$ as the inferred threshold at stopping time $\gT$.
From $\rm{INFER}$, we have
\begin{equation}
\hat{b}_k(t), \delta_{\psi_k}(t) = \rm{INFER}(\phi^t_{X_{k_1}}, \dots, \phi^t_{X_{k_D}})
\label{eqn:conf-seq:compound:inf2}
\end{equation}
We have
\begin{align*}
     \Pr[&\exists t \geq 1, b_k \not\in \hat{b}_k(T)]\\
        &\leq \Pr\left[\bigcup\limits_{i = 1}^{D} \exists t \geq 1,  \neg \phi_{X_{k_i}}^t \right] & \text{(From \ref{eqn:conf-seq:compound:inf2})} \\
      &\leq \sum\limits_{i\in \etB_k}\Pr\left[\exists t \geq 1, \neg \phi^t_{X_{k_i}}\right] & \text{(union bound)} \\
      &= \sum\limits_{i\in \etB_j}\Pr\left[\exists t \geq 1, |\BarE_t[X_{k_i}] - \E_t[X_{k_i}| > \epsilon_{X_{k_i}}(t)\right] & \text{(definition of $\phi^t_{X_{j_i}}$ )} \\
      &\leq \sum\limits_{i\in \etB_j} \delta_{X_{k_i}} & \text{(elementary subexpressions)} \\
      &\leq \delta_{\psi_k} & \text{(applying \ref{eqn:conf-seq:compound:inf1} for $t = \gT$)}
\end{align*}
Thus, $b_{\psi_k}(t)$ is a $1-\delta_{\psi_k}$ confidence sequence for $b_{\psi_k}$
\end{proof}


%The operations that must be overloaded are defined in Figure~\ref{fig:inference}.
%In the grammar described in Figure~\ref{fig:grammar}, we start from pure expressions and eventually create a specification. 
%For pure expressions $<E>$, we can use Theorem~\ref{thm:adaptive-stopping} directly to compute an $(\epsilon, \delta)$ bound for its empirical estimate.
%For each constant $c$, we have $(\epsilon, \delta) = (0, 0)$


\begin{comment}
\subsection{\AVOIRmethodname{}: Correctness of Bounds in Incremental Setting}
For each step, we use the bounds generated from Adaptive Hoeffding inequality with $\delta_X = \Delta$ which ensures correctness. 
However, on successfully finding an optimal solution in subroutine 2, we change $\delta_X$, $\forall X$.
Suppose this change happens at time step $T$, i.e., $\delta_X = \delta^*_T(X)$.
Then, from Theorem~\ref{thm:adaptive-stopping}, we have
\[
     \Pr[|\BarE_T[X] - \E[X]| \leq \epsilon(\delta, T)|] \geq 1 - \delta_T^*(X) 
\]

if 
\[
    \epsilon(\delta_X, n) = \sqrt{ \frac{ \frac{3}{5}\log{(\log_{1.1}{n} + 1)} + \frac{5}{9}\log{(24/\delta_X)} }  {n}}.
\]

From Section~\ref{sec:theoretical:optimization} and the termination criteria for subroutine 2, we have 
\begin{align*}
     \Delta &\geq \sum \delta_i, \delta_i \geq 0 \\
     \implies \Delta &\geq \delta_X\\
     \implies \epsilon(\Delta, n) &= \sqrt{ \frac{ \frac{3}{5}\log{(\log_{1.1}{n} + 1)} + \frac{5}{9}\log{(24/\Delta)} }  {n}} \\
                                  &\geq \sqrt{ \frac{ \frac{3}{5}\log{(\log_{1.1}{n} + 1)} + \frac{5}{9}\log{(24/\delta_X)} }  {n}}
\end{align*}
The bounds generated by subroutine 2 are strictly better than bounds generated using $\Delta$ in subroutine 1, and thus continue to be valid.
In addition, using Theorem~\ref{thm:adaptive-stopping:anytime}, the updated value of $\delta_X$ can be used to provide guarantees for future time steps.
\end{comment}


\section{Optimality}
\label{sec:appendix:optimality}
\begin{proof}
Under $A_\delta$, at the stopping time $\gT^+$,  $\delta^+_i = \Delta/n$, with 
\[
\sum\limits_{i=1}^{n} \delta^+_i = \Delta.
\]
As $\delta_i^+$ are propagated using $\rm{INFER}$ (without constraint rules), we know that they must satisfy the constraints of the optimization problem ~\ref{eq:optimization}. 
%Thus, the constraints of ~\ref{eqn:optimization} are feasible. 
At time $\gT^+$ \AVOIRmethodname{} would find solution $\delta^*_i$ such that minimizes $\sum\limits_{i=1}^{n} \delta_i$.
\begin{align*}
    \sum\limits_{i=1}^{n} \delta^*_i &\leq \sum\limits_{i=1}^{n} \delta^+_i 
                                     = \Delta
\end{align*}
Thus, \AVOIRmethodname{} would terminate at step $\gT^+$, but may find a feasible solution at an earlier step, i.e. $\gT \leq \gT^+$.

\end{proof}

\begin{corollary}
Under mild conditions, \AVOIRmethodname{} terminates in finite steps with an assertion over the required specification.
\label{corollary:termination}
\end{corollary}
\begin{proof}
We know that the stopping time $\gT \leq \gT^+$, the stopping time for \AVOIRmethodname{}.
Thus, \AVOIRmethodname{} would terminate whenever Verifiar can. 
For completeness, we provide the conditions under which Verifair terminates.
Note that $c \in \R$ corresponds to a constant threshold involved in specification, also presented in the grammar and bound proagation rules.
\begin{itemize}
    \item  For every subexpression $C_k$ occurring in the specification such that it is involved in the inverse or inverse constr. rules (i.e., $\BarE[C_k]^{-1}$), $\BarE[C_k] \neq 0$, $C_k \neq 0$
    \item For every subexpression $C_k$ such that it occurs a True/False type inequality (such as $C_k > c$), $\BarE[C_k] \neq c$, $C_k \neq c$
\end{itemize}
\end{proof}

\section{Implementation}
\label{sec:implementation}

\begin{algorithm}
    \caption{\AVOIRmethodname{} Algorithm}
    \label{alg:method}
    \begin{algorithmic}[1] % The number tells where the line numbering should start
        \Require $\Delta$, $\psi$ \Comment{$\Delta$, Specification} 
        \Ensure $T_s$ time step when the value of $\psi$ can be guaranteed with probability $ \geq 1 - \Delta$
        \For{$X_i \in \psi $}
            \State $\delta_{X_i} = \Delta$ \Comment{Set initial value $\forall i$}
            \State $S_{X_i} = 0$ \Comment{Sum of observations}
            \State $n_{X_i} = 0$ \Comment{Number of observations}
        \EndFor
        \State $T = 0$ \Comment{Time step}
        %\State Initialize $N$ as the number of $E$ terms in $\psi$
        \State Initialize $OPT_\psi$ \Comment{Initialize Optimization Problem (Fig.~\ref{fig:inference})}
        \Procedure{Observe}{$X$} 
            \For{$X_i \in X$}
                \State $S_{X_i} = S_{X_i} + X_i$
                \State $n_{X_i} = n_{X_i} + 1$
                \State $\BarE[X_i] = S_{X_i}/n_{X_i}$
                \State Initialize $\delta_{X_i}$ as a symbolic variable
                \State Assign $\epsilon(\delta_{X_i}, n_{X_i})$ symbolic variable
            \EndFor
            \State Propagate $\delta_{X_i}$ using the inference rules
            \State Initialize constraints $g_K$ in $OPT_\psi$ using the computed values
            \State $\delta^*_T = \texttt{Solve}(OPT_\psi)$
            \If{$\delta^*_T \leq \Delta$}
                \State $\delta_{X_i} = \delta^*_T[X_i]$ 
                \State \Return $T_s = T$
            \EndIf
            \State $T = T + 1$
        \EndProcedure
    \end{algorithmic}
    
\end{algorithm}

We built a python library to create specifications that can be implemented as a decorator over decision functions. 
The front end interactive application was implemented using streamlit\footnote{https://streamlit.io/} and the visualizations were built in Vega~\cite{satyanarayan2015reactive}.
Each term in the DSL is implemented through a corresponding python class.
New input/output observations are monitored to update all the terms in a specification.
Inference for evaluating the value and bounds is carried out via operator overloading in these classes. 
%This is fairly straightforward.
In line with previous work~\citep{albarghouthi2017fairsquare,bastani2019probabilistic,albarghouthi2019fairness} on distributional verification, we use rejection sampling for conditional probability estimation.
%In the remainder of this section, we describe how how each of these components are implemented. Further, we describe the process for computing point estimates and probabilistic confidence intervals for the values computed for different terms within the spec.



% Typical fairness constraints require 
%Figure~\ref{fig:grammar} describes the final grammar of our specification DSL.

\subsection{Visual Analysis}
\label{sec:implementation:vis}
Using our specification framework as a backend, we built an interactive application for analysis and refinement of specifications provided in our grammar.
Given a user provided machine learning model, dataset, and specification the application simulates a stream of observations to the provided model.
Following the simulation, a visualization is provided that represents the specification as a syntax tree where each node of the tree corresponds to an element of our grammar.
Figure \ref{fig:casestudy:adult:initial-spec} shows the visualization.

Note that for each observation made by our machine learning model, the specification is evaluated to check for violations.
Each grammar element that makes up the specification is evaluated as well, and thus each grammar element is associated with the value it evaluates to for a given observation.
For specifications \texttt{<spec>}, there is a boolean value associated with each observation, whereas an expectation term, \texttt{<ETerm>}, is associated with a real value.
By selecting one of the nodes in the syntax tree, a user can see a plot of the evaluation values associated with the selected grammar element.
We call these plots evaluation plots and two can be observed at a time 
%(see the plots on the right of figure \ref{fig:casestudy:boston}),
each with shared scales along the horizontal axis which denotes observations over time.
This allows for comparison of multiple grammar elements.
The ability to analyze and compare these evaluation values provides context surrounding specification violations, and assists the user in deciding how to refine a specification.
The case studies in section \ref{sec:casestudy} demonstrate the usefulness of the context provided by these visualizations.


\begin{comment}
The app for interaction with the backend is built using streamlit. It proceeds in multiple stages,
\begin{enumerate}
    \item First, a user selects a dataset of interest. We built support for four datasets, but our framework is generic enough for any arbitrary csv dataset.
    \item Following this choice, the input variables and output variable for a machine learning model must be specified.
    \item A machine learing model is then selected from a dropdown. We provide support for three models. However, this is for demonstration purposes only - the specification is agnostic to the choice of a machine learning model.
    \item Finally, a specification is input by the user of the app. On the press of a button, the model is trained and then evaluated on the selected dataset. The output monitored by the spec is passed off to the Vega module for further analysis.
\end{enumerate}
\end{comment}

\section{AVOIR in Database Setting}

\begin{comment}
\begin{table}
    \centering
    \caption{A summary of the results from the case studies.}
    \label{tab:casestudy:summary}
    \begin{tabular}{ccc}
    \toprule
         Dataset & Setting & Improvement over~\cite{bastani2019probabilistic}  \\
         \midrule
         Adult Income & Database & $10.35\%$ \\
         COMPAS & Materialized View & Interaction\\
         Rate my Profs & ML - BERT & $2.5\%$ \\
         \bottomrule
    \end{tabular}

\end{table}
\end{comment}

In the database literature researchers~\cite{nargesian21tailoring}, have explored an approach to tailoring data integration strategies to ensure that the data set used for analysis has an appropriate representation of relevant (demographic) groups and it meets desired distribution requirements. The authors describe how to acquire such data in an approximate cost-optimal manner for several realistic settings. This work is orthogonal to our work and yet AVOIR can potentially integrate with the authors approach to examine if fairness criteria are being met during the integration process. In other studies on fairness researchers~\cite{yang2018nutritional,asudeh19a,asudeh19b}, have considered the problem of personalized fair ranking functions and discuss approaches to determine if a proposed ranking function satisfies a set of  desired fairness criteria and, if it does not, to suggest modifications that do. AVOIR attempts to solve a  more general purpose problem (not limited to any particular fairness criteria) and is agnostic to the specific model (treats it as a blackbox).  While we have not examined the performance of AVOIR for fair ranking problems, it is something we plan to examine in the future.

To demonstrate how \AVOIRmethodname{} can be integrated within a database system
we use pandas\footnote{https://pandas.pydata.org/} dataframes to simulate the application of \AVOIRmethodname{} in the database setting. 
Specifically, we wrap pandas dataframes with a python `Database' class, and provide a query mechanism to create materialized views.
Queries are provided in the form of python functions that take a dataframe as input and output a corresponding dataframe.
The corresponding view thus generated can be updated with insertion/update/deletion of data.
The specification is added as a decorator inside the refresh function, allowing \AVOIRmethodname{} to track specifications in a database setting.
Note that this tie-in with pandas is only for ease of implementation; the inference engine and optimization can be extended to any database engine.

\section{Additional Case Studies}
\label{sec:appendix:additional-case-studies}
\subsection{Materialized views}
\label{sec:appendix:additional-case-studies:materialized-view}
A materialized view is constructed by querying the dataset to select for employees of the federal government.
We simulate the materialized view using a pandas\footnote{https://pandas.pydata.org/} dataframe wrapped in a python class to monitor updates and run \AVOIRmethodname{} for any monitored specification.

\subsection{Interaction through Vega}
\label{sec:appendix:additional-case-studies:viz}
\begin{figure}[ht!]
    \centering
    \includegraphics[angle=90,width=0.5\textwidth]{avoir/images/adult-spec-tree-initial.png}
    \caption{Tree of initial specification before refinement in the adult income dataset.}
    \label{fig:casestudy:adult:initial-spec}
\end{figure}
Figure~\ref{fig:casestudy:adult:initial-spec} shows a subtree of the specification visualized through Vega.
A developer analyzing this spec can click on the top pink node to see the evolution of the sex fairness part of the specification and superimpose the threshold.
The threshold is set to be evaluated with every 5 new data points added to the materialized view.
Clicking on the corresponding element in the right subtree, the developer can see Figure~\ref{fig:casestudy:adult:specplot:right}. 


\subsection{COMPAS Risk Assessment via Materialized Views}
The Correctional Offender Management Profiling for
Alternative Sanctions (COMPAS) recidivism risk score data is a popular dataset for assessing machine bias of commercial tools used to assess a criminal defendant's likelihood to re-offend.
The data is based on recidivism (re-offending) scores derived from a software released by Northpointe and widely used across the United States for making sentencing decisions.
In 2016, \citet{angwin2016machine} released an article and associated analysis code critiquing machine bias associated with race present in the COMPAS risk scores for a set of arrested individuals in Broward County, Florida over a period of two years.
The analysis concluded that there were significant differences in the risk assessments of African-American and Caucasian individuals.
Northpointe pushed back in a report~\citep{dieterich2016compas} strongly rejecting the claims made by the ProPublica article; instead, they claimed that \citet{angwin2016machine} made several statistical and technical errors in the report.
In this case study, we use \AVOIRmethodname{} to study the claims made by the two aforementioned reports. 
First, we start with the data released by ProPublica and load it into a pandas-simulated DB.
We then create a materialized view that corresponds to the preprocessing steps used in the publicly available ProPublica analysis  notebook\footnote{https://github.com/propublica/compas-analysis}.
We look at ``Sample A''~\citep{dieterich2016compas}, where the analysis of the ``not low'' risk assessments using a logistic regression model reveals a high coefficient associated with the factor associated with race being African-American.
In terms of a fairness metric, this corresponds to false positive rate (FPR) balance (predictive equality)~\citep{verma2018fairness} metrics. 
The associated specification in \AVOIRmethodname{} grammar would be

\begin{lstlisting}[columns=flexible, language=Python]
   E[hrisk | race=African-American & recid=0] / 
   E[hrisk | race=Caucasian & recid=0] < 1.1
\end{lstlisting}

Where \verb|hrisk| is an indicator for high risk assessments made by the \emph{black-box} COMPAS tool as defined by ~\citet{angwin2016machine},  \verb|recid| is an indicator for re-offending within 2 years of first arrest, and a $10\%$-rule is used as the threshold. 
We choose a failure threshold probability of $\Delta = 0.1$, with the optimization run after every batch of $5$ samples.
\AVOIRmethodname{} finds that when the decisions are made in a sequential, online fashion, the assertion for violation of the specification cannot be made with the required failure guarantee.

By analyzing the components using the visualization tool, one can glean that \AVOIRmethodname{} is unable to optimize since the lower FPR in the denominator (FPR for Caucasian individuals) increasing the overall variance and limiting the ability to optimize for guarantees. 
We follow this analysis by using the reciprocal specification, i.e.,
\begin{lstlisting}[columns=flexible, language=Python]
   E[hrisk | race=Caucasian & recid=0] /
   E[hrisk | race=African-American & recid=0] > 0.9
\end{lstlisting}

we find that indeed, the specification is violated with a confidence of over $1 - \Delta = 0.9$ probability, and \AVOIRmethodname{} can detect this violation within about half the number of available assessments (3350 steps) when run in an online setting.
Figure~\ref{fig:casestudy:compas:propublica} demonstrates the progression of the tracked expectation term. 
Thus, if deployed with the corrected specification, \AVOIRmethodname{} would be able to alert Northpointe of a violation/potentially-biased decision making tool.

\begin{figure}[ht!]
    \begin{subfigure}{0.45\linewidth}
        \centering
        \includegraphics[width=\linewidth]{avoir/images/compas-propublica-et.png}
        \caption{(ProPublica) COMPAS, ``Sample A'' False Positive Rate Bias specification required to \emph{above} the $10\% \implies 0.9$ threshold converges to a value that can be guaranteed to be \emph{under} the required threshold.}
        \label{fig:casestudy:compas:propublica}
    \end{subfigure}
    \hfill
    \begin{subfigure}{0.45\linewidth}
        \centering
        \includegraphics[width=\linewidth]{avoir/images/compas-northpointe-et.png}
        \caption{(Northpointe) ``Sample B'' analysis done by Northpointe  using False Discovery Rate that opposed the ProPublica reports.}
        \label{fig:casestudy:compas:northpointe}
    \end{subfigure}
    \caption{COMPAS dataset case study.}
\end{figure}


The Northpointe report~\citep{dieterich2016compas} makes several claims about the shortcomings of this analysis, but one of the primary claims is that \citet{angwin2016machine} used an analysis based on ``Model Errors'' rather than ``Target Population Errors''.
In Fairness metric terms, this refers to the difference between a False Positive Rate (FPR) balance vs False Discovery Rate (FDR) balance, i.e. balancing for predictive parity over predictive equality. 
In probabilistic terms, the difference amounts to comparing $P(\hat{Y} = 1 | Y = 0, g=1, 2)$ (FPR) vs $P(Y = 0 | \hat{Y} = 1, g=1, 2)$ (FDR), where $\hat{Y}$ refers to the decision made by the algorithm, $Y$ refers to the true value, and $g = 1, 2$ reflects group membership~\citep{verma2018fairness}.
This analysis is run on the dataset subset dubbed ``Sample B''.
To test their hypothesis, we run reproduce the corresponding preprocessing steps and run both versions (numerator and denominator being Caucasian) versions of the corresponding specification under the same setup as earlier. 
We find that despite the point estimate being within the required threshold, neither version can be guaranteed with the required confidence with the given data.
Due to paucity of space, we describe only one of the two variants with the corresponding figure (Figure~\ref{fig:casestudy:compas:northpointe}).
\begin{lstlisting}[columns=flexible, language=Python]
   E[recid=0 | race=Caucasian & hrisk] /
   E[recid=0 | race=African-American & hrisk] > 0.9
\end{lstlisting}


We note that the Northpointe report~\citep{dieterich2016compas} does not provide confidence intervals for their claim either. 
Further, even though the report does not release associated code, the point estimates of the False Discovery Rates (FDRs) match those present in the report which increases our confidence in our \AVOIRmethodname{}-based analysis. 

The back and forth exchange has been the subject of much discussion in both academic and journalistic publications~\citep{feller2016computer, washington2018argue}.
Seminal work by \citet{kleinberg2017inherent} proved the impossibility of simultaneously guaranteeing certain combinations of fairness metrics.
While \AVOIRmethodname{} cannot solve this problem, its usage can help provide explicit guarantees on defined metrics.
The specification grammar also provides a simple mechanism for independent replication of claims. 
We conclude this case study by noting that \AVOIRmethodname{} lends itself to successful analysis that is not possible with the Verifair implementation available online. %Importantly, this case study also demonstrates that  \AVOIRmethodname{}, can be directly integrated with database-centric materialized views. 

\begin{comment}
\section{Related work from Databases}
In the database literature researchers~\cite{Nargesian21}, have explored an approach to tailoring data integration strategies to ensure that the data set used for analysis has an appropriate representation of relevant (demographic) groups and it meets desired distribution requirements. The authors describe how to acquire such data in an approximate cost-optimal manner for several realistic settings. This work is orthogonal to our work and yet AVOIR can potentially integrate with the authors approach to examine if fairness criteria are being met during the integration process. In other studies on fairness researchers~\cite{Yang2018,Asudeh19a,Asudeh19b}, have considered the problem of personalized fair ranking functions and discuss approaches to determine if a proposed ranking function satisfies a set of  desired fairness criteria and, if it does not, to suggest modifications that do. AVOIR attempts to solve a  more general purpose problem (not limited to any particular fairness criteria) and is agnostic to the specific model (treats it as a blackbox).  While we have not examined the performance of AVOIR for fair ranking problems, it is something we plan to examine in the future.
\end{comment}

\section{Supported Metrics}
\label{sec:appendix:additional-metrics}

\begin{table}[ht]
    \centering
    \resizebox{\linewidth}{!}{
    \begin{tabular}{lcc}
    \toprule
        Metric Name  & Definition & DSL  \\
    \midrule \\
       Statistical Parity  & \multirow{2}{*}{$\Pr[d=1|G=m] = \Pr[d=1|G=f]$}  & \multirow{2}{*}{$\E[d|G=m] / \E[d|G=f] < c$} \\
       \citep{dwork2012fairness} & &\\
       Predictive Parity & \multirow{2}{*}{$\Pr[Y=1|d=1, G=m] = \Pr[Y=1|d=1, G=f]$} & \multirow{2}{*}{$E[Y=1|d=1, G=f] - \E[Y=1|d=1, G=m] > c$} \\
       \citep{chouldechova2017fair} & & \\
       Equal Opportunity &\multirow{2}{*}{$\Pr[d=0|Y=1, G=m] = \Pr[d=0|Y=1, G=f]$}  & \multirow{2}{*}{$\E[d=0|Y=1, G=m] - \E[d=0|Y=1, G=f] < c$} \\
       \citep{hardt2016equality} & &  \\
       Equalized Odds & $\Pr[d=1|Y=i, G=m] = \Pr[d=1]Y=i, G=f],$ & $(\E[d=1|Y=1, G=f] - \E[d=1|Y=1, G=m] > c_1) \&  $ \\
       \citep{hardt2016equality} & $i = 0, 1$ & $(\E[d=1|Y=0, G=f] - \E[d=1|Y=0, G=m] > c_2)$\\
    \bottomrule
    \end{tabular}
    }
    \caption{Examples of supported metrics.}
    \label{tab:appendix:supported-metrics}
\end{table}

We provide a non-exhaustive list of statistical group-based fairness criteria and show an exact/approximate equivalent in the \AVOIRmethodname{} DSL in Table~\ref{tab:appendix:supported-metrics}.
We use the following notation, adapted from \citet{verma2018fairness}:
\begin{itemize}
    \item[$G$:] Protected or sensitive attribute. For demonstration purposes, we will use the values $m$ and $f$ to denote majority and minorty classes.
    \item[$X$:] Features describing each individual 
    \item[$Y$:] True label for $X$
    \item[$S$:] Probability $\Pr[Y|X, G]$ predicted for a certain class $c$
    \item[$d$:] Predicted decision for $X$, usually derived from $X$
    \item[$c$:] A threshold to test the specification. For ratios based approximations, this would be a number $1 \pm \eps$ for some small $\eps > 0$. For difference based approxiamtions, this number would be some small $\eps > 0$. When multiple terms are present, we use $c_i$ to denote the $i^{\text{th}}$ threshold.
\end{itemize}
\todo{New section}
We assume that the decision function $f$ tracked by \AVOIRmethodname{} as a signature that takes $X, G, Y$ as input and produces $S$ or $d$ as output.
Note that in their python implementation, $=$ would be replaced by \lstinline{==} and $|$ by the \lstinline{given} keyword.



\endinput