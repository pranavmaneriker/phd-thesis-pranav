\newpage \section{Reproducibility}
\label{sec:reproducibility}
To enhance the reproducibility of our work, on the theoretical side, all proofs (with the necessary assumptions) are provided in the appendix. 
Specifically, proofs for the inference engine are in Appendix~\ref{sec:appendix:inference-rules}, and proofs for the correctness of bounds are provided in Appendix~\ref{sec:appendix:confseq}. 
Theorem~\ref{theorem:better-stopping}, which shows how \AVOIRmethodname{} improves over prior work is proved in Appendix~\ref{sec:appendix:optimality}.
To reproduce the results of the case studies in the paper, each case study is encapsulated inside a Jupyter notebook. 
These notebooks are attached along with the source code for \AVOIRmethodname{}.
In addition, all datasets used for generating results for the case studies are also attached in the submitted supplementary documentation.
Finally, the model weights used for the \textit{RateMyProfs} study for exact reproduction are provided in a dropbox folder hosted at  \url{https://www.dropbox.com/sh/n5o4vswnkxv34zr/AABthgLMaYL3MuA0KC39Z1G8a?dl=0}.
\begin{comment}
It is important that the work published in ICLR is reproducible. Authors are strongly encouraged to include a paragraph-long Reproducibility Statement at the end of the main text (before references) to discuss the efforts that have been made to ensure reproducibility. This paragraph should not itself describe details needed for reproducing the results, but rather reference the parts of the main paper, appendix, and supplemental materials that will help with reproducibility. For example, for novel models or algorithms, a link to a anonymous downloadable source code can be submitted as supplementary materials; for theoretical results, clear explanations of any assumptions and a complete proof of the claims can be included in the appendix; for any datasets used in the experiments, a complete description of the data processing steps can be provided in the supplementary materials. Each of the above are examples of things that can be referenced in the reproducibility statement. This optional reproducibility statement will not count toward the page limit, but should not be more than 1 page.
\end{comment}