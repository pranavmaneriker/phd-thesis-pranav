\subsection{Related Work}
\label{sec:related}

\begin{figure}[ht]
    \centering
    \begin{subfigure}[b]{0.47\linewidth}
        \resizebox{\linewidth}{!}{
            \centering
            \newcommand*{\n}{}% Ensure it is not already defined
\newcommand*{\clr}{}% Ensure it is not already defined
\begin{tikzpicture}
                    \begin{axis}[
                            grid=both,
                            xlabel = {$\epsilon$},
                            x tick label style={/pgf/number format/fixed},
                            y tick label style={/pgf/number format/fixed},
                            ylabel = {$\delta$},
                            y label style={rotate=-90},
                            axis lines=left,
                            ymode=log,
                            xmin=0.1,
                            xmax=0.15,
                            legend style={at={(0.03,0.2)},anchor=west},
                            label style={font=\Large}
                    ]
                    \pgfplotsinvokeforeach{600/cb-lilac, 1000/cb-brown}
                    {       
                        \StrBefore{#1}{/}[\n]%
                        \StrBehind{#1}{/}[\clr]%
                        \edef\AddPlot{\noexpand\addplot[color=\clr,smooth,thick, dashed,-,domain=0:0.2] {min(2*\n*\n*exp(-2*\n*x*x), 1)};} %
                        \AddPlot
                        \addlegendentryexpanded{H: $t =\n$};
                        \edef\AddPlot{\noexpand\addplot[color=\clr,smooth,thick,-,domain=0:0.2] {min(24*(( (log10(\n)/log10(1.1)) + 1)^1.08)*(exp(-1.8*\n*x*x)), 1)};} %
                        \AddPlot
                        \addlegendentryexpanded{AH: $t =\n$};
                    }
                    \end{axis}
                    % https://tex.stackexchange.com/questions/325170/parameterizing-color-in-addplot
\end{tikzpicture}
        }
        \caption{At the same concentration $\varepsilon$, lower failure probability $\delta$ for the majority class.}
        \label{fig:n-comparison-hoeffding}
    \end{subfigure}
    \hfill
    \begin{subfigure}[b]{0.48\linewidth}
        \centering
           \begin{grammar}
             <spec> ::= <ETerm>  <comp-op>  c 
                \alt <spec> $\land$ <spec>              
                \alt <spec> $\lor$ <spec> 
        
            <ETerm> ::= $\E[<E>]$
                \alt   $\E[<E>$|$<E>]$
                \alt   c $\in \R$
                \alt   <ETerm> $\ \{+, -, \times, \div \}\ $ <ETerm>
\end{grammar}
         \caption{ $\langle E \rangle$ refers to pure expressions and $\langle comp-op \rangle$ is any comparison operator $\in \{>, <, =, \neq\}$.}
         \label{fig:grammar}
    \end{subfigure}
    \caption{\figleft{} Failure probability of Bernoulli r.v. being concentrated around its mean for different $n$. H = (online) Hoeffding, AH = Adaptive Hoeffding. \figright{} Modified Grammar for Specification.}
\end{figure}

There are a plethora of fairness criteria and subtle changes in their definition can change the implications on decision making~\citep{castelnovo2021zoo}.
Practitioners need support when selecting, designing, and guaranteeing fairness for deployed machine learning algorithms.
Prior work on fairness has helped develop nuanced notions and algorithms to help train more `fair' machine learning models.
These include group fairness measures such as, inter alia, minimizing disparate impact~\citep{calders2009building,feldman2015certifying}, maximizing the equality of opportunity~\citep{hardt2016equality}
In contrast with group fairness notions, causal notions of fairness~\cite{kusner2017counterfactual} and individualized notions of fairness~\cite{dwork2012fairness} provide alternative statistical mechanisms for understanding discriminatory behaviors of automated decision systems.
\citet{thomas2019preventing} proposed the Seldonian Framework as a generic mechanism for model users to design algorithms that help train machine learning models that can regulate them against undesirable behaviors.

We focus on the problem of detecting and diagnosing whether systems designed under any framework follow any prescribed regulatory constraints that are supported within the grammar of \AVOIRmethodname{}.
\todo{More nuanced comparison}
That is, we are agnostic to the design itself; rather, we are interested in testing the adherence of models to specified criteria.
We use a probabilistic framework to verify this behavior.
Alternative frameworks such as the AI Fairness 360~\citep{bellamy2019AI}  provide mechanisms to quantify fairness uncertainty, though they are restricted to pre-supported metrics.
Uncertainty quantification~\citep{ghosh2021uncertainty,ginart2022mldemon} is an alternative mechanism to provides adaptive guarantees, however, existing work is designed for commonly used outcome metrics such as accuracy, F1-score, etc., rather than for fairness metrics. 
Justicia~\citep{ghosh2021justicia} optimizes uncertainty for fairness metrics estimates using stochastic SAT solvers but can only be applied to a limited class of tree-based classification algorithms.

Prior work on fairness testing is most closely related to \AVOIRmethodname{}.
Fairness testing~\citep{galhotra2017fairness} provides a notion of causal fairness and generates tests to check the fairness of a given decision-making procedure.
Given a specific definition of fairness, Fairtest~\citep{tramer2017fairtest} and Verifair~\citep{bastani2019probabilistic} build a comprehensive framework for investigating fairness in data-driven pipelines. 
%By auditing the system's fairness, a more balanced rule can be generated for future use.
Fairness-aware Programming (FP) \citep{albarghouthi2019fairness} combined the two demands of machine learning testing and fairness auditing to make fairness a first-class concern in programming. 
Fairness-aware programming applies a runtime monitoring system for a decision-making procedure with respect to an initially stated fairness specification.
The overall failure probability of an assertion is computed as the sum of failure probabilities of each constituting sub-expression (using the union bound).
FP does not provide any specific mechanism for splitting uncertainty, and Verifair splits it equally across all constituent \textit{elementary subexpressions}.
Thus, assertion bounds for subexpressions in both FP and Verifair are split inefficiently. 
Proving guarantees for overall uncertainty across multiple groups can be ameliorated by balancing it across subexpressions with differences in the number of observed samples.
For example, consider Bernoulli r.vs  $X_{1,2}$ for which we derive concentration guarantees $\Pr[|\E[X_i] - \BarE[X_i]| \geq \epsilon_{i}] \leq \delta_{i}$ after $t_{i}$ observations. 
From the Hoeffding inequality, $\delta = 2e^{-2t\epsilon^2}$.
\todo{Clarify the concentration.}
We can claim tighter guarantees for $X_2$ if $t_2 > t_1$ as the failure probability is lower at the same concentration $\eps$.
That is, $\eps_1 = \eps_2, t_2 > t_1 \implies \delta_1 > \delta_2$.
Varying $\eps$ across subexpressions to minimize the overall $\delta = \delta_1 + \delta_2$ allows convergence in fewer iterations.
This observation motivates us to optimize over sub-expressions and provide tighter overall concentration for compound expressions.
%Figure~\ref{fig:n-comparison-hoeffding} demonstrates this for $n = \left<200, 1000\right>$.
Adaptive versions of these inequalities also have similar patterns (see Figure~\ref{fig:n-comparison-hoeffding}).
