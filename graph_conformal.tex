\chapter{Uncertainty Quantification and Fairness in Graph Structured Data}
In the previous chapter, 
Conformal prediction has become increasingly popular as a method for quantifying the 
uncertainty associated with machine learning models. 
The computational efficiency of the inductive approach has made it the method du jour for building new methods for conformal prediction.
Recent work in graph uncertainty quantification has built upon this approach for conformal prediction on graphs.
The nascent nature of these explorations has lead to conflicting choices for implementations baselines and evaluation of approaches.
We critically analyze the choices made and describe the tradeoffs associated with existing graph conformal prediction work. 
Our theoretical and empirical results provide the rationale for our recommendations for future scholarship in graph conformal prediction.