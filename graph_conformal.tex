\chapter{Conformal Prediction for Graph Structured Data}
\label{chp:graphConformal}
The previous chapter used confidence intervals to generate online, anytime-valid bounds for the outputs associated with different decision-making models.
However, confidence intervals require a strong assumption about the data-generating distribution at test time, i.e., independent and identically distributed (IID) data.
For graph-structured data, the edges between different nodes denote potential dependencies between the nodes.
Thus, the IID assumption is violated, and the corresponding confidence intervals are no longer a viable option.
\emph{Conformal prediction} is a method that provides valid confidence sets/intervals for graph-structured data under a weaker assumption - exchangeability.
While the estimates generated by conformal prediction are no longer online or anytime-valid, the associated guarantees can provide a foundation for understanding the uncertainty associated with the predictions in graph-structured data.
This chapter will discuss the theoretical underpinnings of conformal prediction and the tradeoffs associated with its application to graph-structured data.

Conformal prediction has become increasingly popular for quantifying the uncertainty associated with machine learning models. 
The computational efficiency of the split conformal prediction approach has made it the method du jour for exploring new approaches for quantifying the uncertainty of predictions from large, computationally expensive machine learning models.
Recent work in graph uncertainty quantification has built upon this approach for conformal prediction on graphs.
The nascent nature of these explorations has led to conflicting choices for implementations, baselines, and evaluation of approaches.
We critically analyze the choices made and describe the tradeoffs associated with existing graph conformal prediction work. 
Our theoretical and empirical results provide the rationale for our recommendations for future scholarship in graph conformal prediction.

\section{Introduction}
Modern machine learning models trained on losses based on point predictions are prone to be overconfident in their predictions~\citep{guo2017calibration}. 
The Conformal Prediction (CP) framework~\citep{vovk2005algorithmic} provides a mechanism for generating statistically sound post hoc prediction sets (or intervals, in case of continuous outcomes) with coverage guarantees under mild assumptions.
The usual assumption made in CP is that data are exchangeable, i.e, the joint distribution of the data is invariant to permutations of the data points.
The guarantees provided by CP are distribution-free, and can be added post hoc to arbitrary, black-box predictor scores.
This makes CP an ideal candidate for quantifying uncertainty in complex models, such as neural networks.

Network-structured data such as social networks, transportation networks, and biological networks are ubiquitous in modern data science applications.
Graph Neural Networks (GNNs) have been developed to model vector representations of such network-structured data, and have been shown to be effective in a variety of tasks such as node classification, link prediction, and graph classification~\citep{hamilton2020graph, wu2022graph}.
Uncertainty quantification approaches built for independent and identically distributed (iid) data cannot directly be applied to graph data, as the network structure introduces dependencies between the data points.
However, recent work~\citep{clarkson2023distribution,zargarbashi23conformal,huang2024uncertainty} has demonstrated that in certain settings, CP can be applied to graph data to generate statistically sound prediction sets for the node classification task.

Variations of CP include full CP~\citep{vovk2005algorithmic} which has significant computational cost as the score function must be recomputed with replacement for each data point within the calibration set.
Additionally, cross-conformal prediction~\citep{vovk2015cross}, CV+/Jackknife+~\citep{barber2021predictive} are other variations of CP which are computationally more efficient than full CP, but less efficient than split CP.
Prior work in CP on graphs has mainly focused on the split CP setting due to its computational efficiency, ease of implementation, and distribution-free guarantees with black-box models. 
We focus on split CP in this work.
There is a lack of consensus for the choice and setup of baselines, splitting of common datasets, and evaluation metrics for methods.
In this work, we aim to analyze the choices made by existing work and understand the trade offs associated with these choices.
%In addition, we create a python library which implements different variations of these approaches which would help standardize practices in the evaluation of CP for graph data.


\section{Conformal Prediction}
Conformal prediction is used to quantify the uncertainty of a model by providing prediction sets/intervals with coverage guarantees.
We will focus on conformal prediction in the classification setting.
Given a calibration dataset $\calib = \{(\vx_i, y_i)\}_{i=1}^n$, where $\vx_i \in \gX = \R^d$ and $y_i \in \gY = \{1, \dots, K\}$, conformal prediction can be used to construct a prediction set $C$ such that
\begin{align*}
    \Pr\left[y_{n+1} \in C(\vx_{n+1}) \right] \geq 1 - \alpha
\end{align*}
where $1 - \alpha \in [0, 1]$ is a user-specified coverage level.
The only assumption required for the coverage guarantee is that $\calib \cup \{(\vx_{n+1}, y_{n+1})\}$ is exchangeable.
The following theorem provides a general recipe for constructing a prediction set with coverage guarantee.
\begin{theorem}[\citet{vovk2005algorithmic}]
    Suppose $\{(\vx_i, y_i)\}_{i=1}^{n+1}$ are exchangeable, $s: \gX \times \gY \rightarrow \R$ is a score function measuring the non-conformity of $(\vx, y)$, with higher scores indicating lower conformity, and a target $\alpha \in [0, 1]$. % is a target significance level.
    Let $\hat{q}(\alpha) = \text{Quantile}\left(\frac{\ceil{(n+1)(1-\alpha)}}{n}; \{s(\vx_i, y_i)\}_{i=1}^{n}\right)$
    Define $C_{\alpha}(X) = \{y \in \gY: s(\vx, y) \leq \hat{q}(\alpha)\}$.
    Then,
    \begin{align}
        1 - \alpha +\frac{1}{n+1} \geq \Pr\left[y_{n+1} \in C_{\alpha}(\vx_{n+1}) \right] \geq 1 - \alpha
        \label{eq:CP:coverage}
    \end{align}
    \label{thm:CP:coverage}
\end{theorem}

While Theorem~\ref{thm:CP:coverage} does not place any restrictions on the choice of the score function, this choice can have a significant impact on the size of the prediction set.
$s$ is usually called the non-conformity score function and measures the degree of non-agreement between the input $\vx$ and the label $y$, given exchangeability with the calibration data $\calib$ i.e., larger scores indicate worse agreement between $\vx$ and $y$.
Note that the setup of theorem~\ref{thm:CP:coverage} is called split CP, as the score function remains fixed for the calibration split.
In other versions of CP, the score function is usually more expensive as it maps $\gX^k \times \gY^k \rightarrow \R$, for some $k \in \N$ which varies between $n$ for full conformal prediction and smaller values for cross-conformal prediction and CV+/Jackknife+.
In split CP, the dataset is partitioned as $\mathcal{D} = \gD_{\text{train}}\cup \gD_{\text{valid}}\cup \gD_{\text{calib}}\cup  \gD_{\text{test}}$.

%Further, it only provides a marginal coverage guarantee.
%\pmcomment{discuss the difference between marginal and conditional coverage and requirements for conditional coverage. Additionally, discuss beta distribution?}



%\section{Applying Conformal Prediction to Graph Structured Data}
\section{Node Classification and Conformal Prediction in Graphs}
The usual tasks of interest in graph data are node classification, link prediction, and graph classification. 
In this work, we focus on node classification and its extensions to conformal prediction.
Consider an attributed homogeneous graph $\gG = (\gV, \gE, \mX)$, where $\gV$ is the set of nodes, $\gE$ is the set of edges and $\mX$ is the set of node attributes.
Let $\mA$ denote the adjacency matrix for the graph.
Further, let $\gY = \{1, \dots, K\}$ denote set of class labels associated with the nodes.
For $v \in \gV$, $\vx_v \in \R^d$ denotes the features and $y_v \in \{1, \dots, K\}$ denotes the corresponding class label.
The task of node classification is to learn a model $F: \gX \to Y$ which predicts the label for each node given node features as input.
Corresponding to the CP partitions, we denote the nodes in the training set as $\gV_{\text{train}}$, validation set as $\gV_{\text{valid}}$, calibration set as $\gV_{\text{calib}}$, and test set as $\gV_{\text{test}}$.
We denote $\gV_d = \gV_{\text{train}} \cup \gV_{\text{valid}}$ 
as the development set of the base model (non-conformalized). 
Note that labels are available only for nodes in the train, validation and calibration sets, and must be predicted for the test set.
Next, we discuss the different settings for node classificaiton in graphs and the applicability of conformal prediction.

\noindent \textbf{Transductive setting}
In this setting, the model has access to the fixed graph $\gG$ during training, validation, calibration, and testing.
However, the labels associated with the test nodes $\gD_{\text{test}}$ are unknown. 
%We assume that $\gV_{\text{test}} \cap \gV_{\text{calib}}$ are exchangeable.
We designate a fixed set of nodes disjoint from the training and validation set as  $\gV_{\text{test}} \cap gV_{\text{calib}}$ and then randomly sample nodes from this set to form $\gV_{\text{calib}}$ and $\gV_{\text{test}}$.
This is the setting considered in~\citet{zargarbashi23conformal} and~\citet{huang2024uncertainty}.
Note that the labels for the calibration nodes are not available during training/validation, though the neighborhood information $(\gV, \gE)$ and the features $\vx_v$ and labels $y_v$, $v \in \gV_d$ are available.
During the calibration phase, the features and labels for the calibration nodes, along with the neighborhood information, are used to compute the non-conformity scores.
This split ensures that the base model cannot distinguish between the calibration and test nodes, and hence exchangeability holds for $v \in \gV_{\text{calib}}$ and $\gV_{\text{test}}$.

\noindent \textbf{Inductive setting}
We briefly describe the inductive setting and note that the exchangeability assumption will be violated in this setting.
The base model is provided with the graph induced by the development nodes only $(\gV_d, \gE_d, \mX_d)$.
In the calibration/test phases, the nodes arrive either one at a time or in batches.
Thus, nodes arriving later in the sequence will have access to neighbors that arrived earlier, breaking the exchangeability assumption.

Thus, following in line with previous work, we focus on the transductive setting.

The following theorem shows that in the inductive setting, a score model trained on the combination of the calibration set will generate scores exchangeable with the test set, thus allowing the use of conformal prediction in the transductive setting.

\begin{theorem}[\citet{zargarbashi23conformal,huang2024uncertainty}]
    Let $\gG = (\gV, \gE, \mX)$ be an attributed graph, and $\gV_{\text{calib}} \cup \gV_{\text{test}}$ be exchangeable.
    %Let $F: \gX \to \Delta_{\gY}$ be a model trained on $\gV_{\text{train}} \cup \gV_{\text{valid}}$.
    %Let $\hat{F}: \gX \to \Delta_{\gY}$ be a model trained on $\gV_{\text{calib}}$.
    %Then, the scores $\hat{F}(\vx)$ are exchangeable with $F(\vx)$ for $\vx \in \gV_{\text{test}}$.
    \label{thm:exchangeability}
\end{theorem}
\pmcomment{TODO: thm and proof for all scores}

%\pmcomment{Exchangeability for transductive, and potential for inductive}

%\subsubsection{Node Classification}
%Let $f: \gX \to \R_K$ denote the class wise scores associated with a node classification model trained on a separate split of the data ($\gD_{\text{train}}$).
%For example, these could be the pre-final output layer of a graph neural network (either before or after softmax normalization).
%and a trained model with classwise prediction the goal is to learn a model $\pi: \gX \to \Delta_K$, where $\Delta_K$ is the $K$-dimensional probability simplex.
%
%\pmcomment{below text optionol}
For the following sections, we will assume that the base model $\hat{\pi}: \gX \to \Delta_{\gY}$, where $\Delta_{\gY}$ is the probability simplex over the elements of $\gY$, is learned using the training and validation sets $\parens{\gD_{\text{train}}}$ and $\parens{\gD_{\text{valid}}}$. 
The calibration set $\parens{\gD_{\text{calib}}}$ is used to determine the $\hat{q}(\alpha)$ from Theorem \ref{thm:CP:coverage} and the test set $\parens{\gD_{\text{test}}}$ is the set for which we want to compute our prediction sets.
In general, we don't need the scores to lie over a simplex; they can be in $\R^K$.
However, this greatly simplifies the exposition for the following sections and is the standard practice in prior work.



\section{Conformal Scores for Graphs: Choices and Trade-offs}
\label{chp:graphConformal:sec:conformal_scores_tradeoffs}
In this section, we critically examine some decision made in the implementations of existing graph conformal prediction work.
We discuss the tradeoffs associated with these choices and provide recommendations for future scholarship in graph conformal prediction.

\subsection{Dataset splits and Training}

 A base model $\pi: \gX \to \Delta_{\gY}$, where $\Delta_{\gY}$ is the probability simplex over the elements of $\gY$, is learned using the training and validation sets $\parens{\gD_{\text{train}}}$ and $\parens{\gD_{\text{valid}}}$. The calibration set $\parens{\gD_{\text{calib}}}$ is used to determine the $\hat{q}(\alpha)$ from Theorem \ref{thm:CP:coverage} and the test set $\parens{\gD_{\text{test}}}$ is the set for which we want to compute our prediction sets.
There are several different methods of partitioning the data to get these different sets. Two methods which are used in other works on graph conformal prediction for classification are (1) full-split paritioning~\cite{huang2024uncertainty} and (2) label-based sample partitioning~\cite{zargarbashi23conformal}.
%\avcomment{Add citations for methods that use each one} %These methods originate from works that consider the classification task in either a supervised or semi-supervised setting, respectively.
 
\noindent \textbf{Full-Split Paritioning}
In such a case, the choice of data split can be done such that each subset of the parition adheres to a size constraint. For example, in CF-GNN \cite{huang2024uncertainty} the authors split the datasets in their experiments randomly, but adhering to a $20/10/70$ split of $\gD_{\text{train}}, \gD_{\text{valid}},$ and $\gD_{\text{calib}}\cup \gD_{\text{test}}$....

\noindent \textbf{Label-Based Sample Paritioning}
In the case of 

Suppose all of the labeling information is not given. This can be the case if the GNN is trained through semi-supervised learning or if an inductive learning setting is simulated....



\subsection{On TPS and Adaptability}
Threshold Prediction Sets (TPS)~\citep{sadinle2019least} is a simple technique for generating conformal prediction sets.
The score function $s(\vx, y) = 1 - \hat{\pi}(\vx)_y$ directly maps the probability from the base model for the correct class into a non-conformity score.
The score is higher if the model has a lower probability assigned to the correct class, indicating the label is less conforming with the model.
A $1-\alpha$ (approximate) quantile creates a probability inclusion threshold for this score over the calibration set ensures coverage and can be shown to generate prediction sets with the smallest expected size~\cite{sadinle2019least}.
However, the TPS score has been known to undercover hard examples and overcover easy ones~\citep{angelopoulos2021uncertainty,zargarbashi23conformal} to achieve this efficiency.
Here, hard/easy refers to the coverage achieved by the prediction set in relation to the prediction set size.
By overcovering easy examples, TPS can still maintain the overall coverage guarantee without having to correctly account for coverage over harder examples.

We note that this discrepancy is claimed to occur as the TPS scores are not `adaptive', and consider only one dimension of the score for each calibration sample.
However, \citet{sadinle2019least} also proposed a labelwise control version of TPS.
Instead of defining a single threshold for all classes, they separately compute the threshold for each class for a corresponding $\alpha$.
Thus, we define classwise quantile thresholds as
\[
    \hat{q}(\alpha, y_j) = \text{Quantile}\left(\frac{\ceil{(n+1)(1-\alpha)}}{n}; \braces{s(\vx_i, y_i)\, {i=1, \dots, n, y_i = y_j}}\right)
\]
and the corresponding prediction sets as
\[
    C_{\text{TPS}}(\vx) = \{y \in \gY: s(\vx, y) \leq \hat{q}(\alpha, y)\}
\]
Note that this version would provide coverage for each class label, making it more `adaptive'.
The version defined by \citet{sadinle2019least} allows controlling $\alpha_y$ for each class label, though we set $\alpha_y = \alpha$ for label-adaptability.
The trade-off here is that we have fewer calibration samples used for each quantile threshold dimension, which may lead to higher variance in the distribution of coverage~\cite{vovk2012conditional}.
We call this variation of TPS as TPS-Classwise, and consider it in our baselines for comparison.

\subsection{APS and Randomized Sets}
The most popular baseline in work on graph conformal prediction is adaptive prediction sets (APS). 
%\pmcomment{TODO: Include a bunch of papers that use APS as a baseline}
\citet{romano2020classification} introduce APS by defining an optimal prediction set construction mechanism under oracle probability.
Suppose we estimate a prediction function $\hat{f}$ that correctly models the oracle probability $\Pr[Y=y|X_{test}=\vx] = \pi_y(\vx)$ for each $y \in \gY = \{1, \dots, K\}$ 
Let $\pi_{(1)}(\vx), \dots, \pi_{(K)}(\vx)$ be the sorted probabilities in descending order.
For any $\tau \in [0, 1]$, define the generalized conditional quantile funciton at $\tau$ as
\begin{align}
    L(x; \pi, \tau) =  \min\left\{ k \in \{1, \dots, K\}, \sum\limits_{j=1}^k \pi_{(j)}(\vx) \geq \tau \right\}
    \label{eq:APS:L}
\end{align}
Then the corresponding prediction set, $C_\alpha^{\text{or}}(\vx)$ can be constructed from the probabilities needed to reach $1-\alpha$ coverage.
\[
    C_\alpha^{\text{or+}}(\vx) = \{y \in \gY: \pi_y(\vx) \geq \pi_{(L(\vx; \pi, 1-\alpha))}(\vx)\}
\]
where $\text{or}$ indicates the usage of the oracle probability.
Further, they define tighter prediction sets in a randomized fashion using an additional uniform random variable $u \sim \text{Uniform}(0, 1)$ as a parameter to construct a generalized inverse. 
This idea draws upon the idea of uniformly most powerful tests in the Neyman-Pearson lemma for level-$\alpha$ sets~\cite{neyman1933ix}.
%\pmcomment{cite, potentially link to Karlin-Rubin}. 
Define
%\pmcomment{$\leq$ vs $<$ and its effect on proof}
\begin{align}
    S(\vx, u; \pi, \tau) = \begin{cases}
        \{y \in \gY: \pi_y(\vx) > \pi_{(L(\vx; \pi, \tau))}(\vx)\} & u < V(\vx; \pi, \tau) \\
        \{y \in \gY: \pi_y(\vx) \geq \pi_{(L(\vx; \pi, \tau))}(\vx)\} & \text{otherwise}
    \end{cases}
    \label{eq:APS:S}    
\end{align}
i.e., the class at the $L(\vx; \pi, \tau)$ rank is included in the prediction set with probability $1 - V(\vx; \pi, \tau)$, where
\[
V(\vx; \pi, \tau) = \frac{1}{\pi_{(L(\vx; \pi, \tau))}(x)} \left\{ \left[\sum\limits_{j=1}^{L(\vx; \pi, \tau)}{\pi_{(j)}(x)} \right] - \tau \right\}
\]
The corresponding randomized prediction sets are $C_\alpha^{\text{or}}(\vx) = S(\vx, U; \pi, 1 - \alpha)$, $U \sim U(0, 1)$
Note that in general, the coverage guarantees provided in conformal prediction hold only in expectation over the randomness in $(\vx_i, y_i), i = 1, \dots, n+1$.
The randomized prediction sets continue to provide the guarantee with additional randomness over $u_i$.
To make this work for a non-oracle probability $\hat{\pi}(\vx)$, they define a non-conformity score $A$
\begin{align}
    A(\vx, y, u;\hat{\pi}) = \min\{\tau \in [0, 1]: y \in S(\vx, u; \hat{\pi}, \tau)\}
    \label{eq:APS:score}
\end{align}

Assume that $\hat{\pi}$ are all distinct - for ease of defining rank.
Suppose the rank of the true class amongst the sorted $\hat{\pi}$ be $r_y$, i.e., $\sum\limits_{i=1}^K \1[\hat{\pi}_i(\vx) \geq \hat{\pi}_{y}] = r_y$
Solving for $\tau$ as a function of $\hat{\pi}$  (see Appendix~\ref{appx:APS:tau}, for proof),
\begin{align}
A(\vx, y, u;\hat{\pi}) = \left[ \sum\limits_{i=1}^{r_y} \hat{\pi}_{(i)}(\vx) \right] - u \hat{\pi}_{y}
\end{align}

Instead, if a deterministic set is used to define the conformal score instead (i.e., the randomized set construction is not carried out), then we could just add the probabilities until the true class is included:
\begin{align}
    \Tilde{A}(\vx, y;\hat{\pi}) = \left[ \sum\limits_{i=1}^{r_y} \hat{\pi}_{(i)}(\vx) \right]
\end{align}
This version of APS still provides the same conditional coverage guarantees and has a simpler exposition as the prediction sets are constructed by greedily including the classes until the true label is included.
 Thus, this version is provided as the implementation in the popular monographs on conformal prediction by~\citet{angelopoulos2021gentle, angelopoulos2023conformal}.
However, the lack of randomization may sacrifice on the efficiency. 
This modification of score affects both the quantile threshold computation during the calibration phase and the prediction set during the test phase.
We will now show the conditions that impact the efficiency more formally.
Let 
\[
    \hat{q}_{A} = \text{Quantile}\left(\frac{\ceil{(n+1)(1-\alpha)}}{n}; \{A(\vx_i, y_i, u_i; \hat{\pi})\}_{i=1}^{n}\right)
\]
and
\[
    \hat{q}_{\Tilde{A}} = \text{Quantile}\left(\frac{\ceil{(n+1)(1-\alpha)}}{n}; \{\Tilde{A}(\vx_i, y_i; \hat{\pi})\}_{i=1}^{n}\right)
\]
Define $A_i(y) := A(\vx_i, y, u_i; \hat{\pi})$ and $\Tilde{A}_i(y) := \Tilde{A}(\vx_i, y, u_i; \hat{\pi})$
From the definition of the prediction sets and non-conformity scores, we have 
\[
    C_A(\vx_{n+1}) = \{y \in \gY: A_{n+1}(y) \leq \hat{q}_{A}\}
\] and 
\[
    C_{\Tilde{A}}(\vx_{n+1}) = \{y \in \gY: \Tilde{A}_{n+1}(y) \leq \hat{q}_{\Tilde{A}}\}
\] 
denote the prediction sets corresponding to the two score functions (with and without randomization).
Define $C_{A}^{i} = C_{A}(\vx_{i})$. 
Let $y'_i \in \{1, 2, \dots, K\} \setminus \{y_i\}$ be any incorrect class label for each $\vx_i$.
Define 
\[
    \alpha_c^A \in [0, 1], \hat{q}_A = \text{Quantile}\left( \frac{\ceil{(n+1)(1 - \alpha_c^A)}}{n}; \{A(\vx_i, y'_i, u_i; \hat{\pi})\}_{i=1}^n \right)
\]
\[
    \alpha_c^{\Tilde{A}} \in [0, 1], \hat{q}_{\Tilde{A}} = \text{Quantile}\left( \frac{\ceil{(n+1)(1 - \alpha_c^{\Tilde{A}})}}{n}; \{A(\vx_i, y'_i, u_i; \hat{\pi})\}_{i=1}^n \right)
\]    
as the thresholds for which the corresponding quantile of the scores for the correct classes $A_i(y_i)$ and $\Tilde{A}_i(y_i)$ achieve $1-\alpha$ coverage.
Then from the exchangeability of $A(\vx_i, y'_i, u_i; \hat{\pi})$
\[
    1 - \alpha_c^A \leq \Pr[y'_{n+1} \in C_{A}^{n+1}] \leq 1 - \alpha_c^A + \frac{1}{n+1}
\]
and similarly, from the exchangeability of $\Tilde{A}(\vx_i, y'_i, u_i, \hat{\pi})$
\[
    1 - \alpha_c^{\Tilde{A}} \leq \Pr[y'_{n+1} \in C_{\Tilde{A}}^{n+1}] \leq 1 - \alpha_c^{\Tilde{A}} + \frac{1}{n+1}
\]
We will show that as long as these thresholds are sufficiently separated, the randomized prediction set will be more efficient than the non-randomized one. 

\begin{theorem}
Assume that $\alpha_c^A - \alpha_c^{\Tilde{A}} \geq \frac{2}{n+1}$ then prediction set constructed using randomization is more efficient than without. Formally, 
\[
    \E\left[|C_{\Tilde{A}}(\vx_{n+1})| - |C_A(\vx_{n+1})|\right]  \geq 0
\]   
\label{them:APS:efficiency}
\end{theorem}
\begin{proof}
Consider the case with only two potential class labels $K = \{1, 2\}$. 
%Let $\Pr[y_{n+1} = 1] = \eta = 1 - \Pr[y_{n+1} = 2]$.

%From the definition of $C_A^i$
%\[
%    \alpha_c = \min\left\{\alpha \in [0, 1] : \hat{q}_A \leq \text{Quantile}\left( \frac{\ceil{(n+1)(1 - \alpha)}}{n}; \{A(\vx_i, y'_i, u_i; \hat{\pi})\}_{i=1}^n \right)\right\}
%\]    
%\pmcomment{Such an $\alpha$ may not exist.}
We have
\begin{align*}
    \E\left[|C_{A}^{n+1}|\right] &= \E\left[\sum\limits_{i=1, 2}\1[i \in C_{A}^{n+1})]\right] \\
                                 &= \E\left[\1[y_{n+1} \in C_{A}^{n+1})]\right] + \E\left[\1[y'_{n+1} \in C_{A}^{n+1})]\right]  & \text{linearity}\\
                                 &= \Pr[y_{n+1} \in C_{A}^{n+1}] + \Pr[y'_{n+1} \in C_{A}^{n+1}] & \text{$\E[\1[A]] = \Pr[A]$}\\
                                 &\leq 1 - \alpha + 1 - \alpha_c^A + \frac{2}{n+1} & \text{(Exchangeability, Theorem~\ref{thm:CP:coverage})} 
\end{align*}
From a similar argument, we can show that 
\[
    \E\left[|C_{\Tilde{A}}^{n+1}|\right] \geq 1 - \alpha + 1 - \alpha_c^A
\]
Thus, 
\begin{align}
    \E\left[|C_{\Tilde{A}}^{n+1}| - |C_{A}^{n+1}|\right] &\geq 1 - \alpha + 1 - \alpha_c^{\Tilde{A}} - \left(1 - \alpha + 1 - \alpha_c^A + \frac{2}{n+1} \right)\\
    &= \alpha_c^A - \alpha_c^{\Tilde{A}} - \frac{2}{n+1}
\end{align}
which is equivalent to our assumption, and this completes the proof.
For $K$ classes, 
\begin{align*}
    \E\brackets{\abs{C_{A}^{n+1}}} &= \Pr[y_i \in C_{A}^{n+1}] + (K-1)\sum\limits_{y'_i} \Pr[y'_i \in C_{A}^{n+1}] \\
\end{align*}
Thus, 
\begin{align*}
    \E\brackets{\abs{C_{A}^{n+1}}}&\leq 1 - \alpha + \frac{1}{n+1} + (K-1)\left(1 - \alpha_c^A + \frac{1}{n+1}\right) \\
     &= 1 - \alpha + (K-1)\left( 1 - \alpha_c^A\right) + \frac{K}{n+1}
\end{align*}
and 
\[
    \E\brackets{\abs{C_{A}^{n+1}}} \geq 1 - \alpha + (K-1)\left(1 - \alpha_c^A\right)
\]
similar bounds can be derived for $\E\brackets{\abs{C_{\Tilde{A}}^{n+1}}}$.
Thus, 
\begin{align*}
    \E\left[|C_{\Tilde{A}}^{n+1}| - |C_{A}^{n+1}|\right] &\geq (K-1)\left(\alpha_c^A - \alpha_c^{\Tilde{A}}\right) - \frac{K}{n+1} \\
    &\geq (K-1) \left(\alpha_c^A - \alpha_c^{\Tilde{A}} - \frac{K}{(K-1)(n+1)}\right)\\
    & >  (K-1)\left(\alpha_c^A - \alpha_c^{\Tilde{A}} - \frac{2}{n+1}\right) \geq 0
\end{align*}
Which completes the proof in the general case.

\end{proof}
Intuitively, as each score in $A$ gets shifted by a small $u\pi$ term to the left, $q_A$ would be lower than $q_{\Tilde{A}}$.
Thus, the significance levels that we would search for in the complementary scores $1-\alpha_c^A$ would be less than $1-\alpha_c^{\Tilde{A}}$.
$1 - \alpha_c^A < 1 - \alpha_c^{\Tilde{A}} \implies \alpha_c^A - \alpha_c^{\Tilde{A}} > 0$.
If the shift is sufficiently large, then the randomized prediction set will be more efficient than the non-randomized one.
%For the assumption to hold, the shifts $u\pi'$ in the complementary scores $A'$ of the incorrect classes should be smaller to ensure that the $\alpha_c^A$ is larger than $\alpha_c^{\Tilde{A}}$.
%For a good classifier, $\pi > \pi'$ in general. 
In Figure~\ref{fig:APS:efficiency}, we show what this looks for a practical example dataset and classifier.
In the plot on the bottom, the (normalized) sorted index at which the lower threshold $q_A$ is reached over the scores $A'$ is lower, i.e., $1 - \alpha_c^A$ is lower, and hence $\alpha_C^A$ is higher.
Note that the dependence on $\frac{1}{n+1}$ indicates that the improvements would be more pronounced for larger $\calib$.
\pmcomment{todo show this}

\begin{figure}
    \centering
    \begin{subfigure}{0.7\linewidth}
        \centering
        \includegraphics[width=\linewidth]{graphConformal/figures/aps_dist}
    \end{subfigure}
    \begin{subfigure}{0.7\linewidth}
        \centering
        \includegraphics[width=\linewidth]{graphConformal/figures/aps_sorted}
    \end{subfigure}
    \caption{Figure showing the scores for an example dataset. (top) shows the shift in the quantile for $A$ and $\Tilde{A}$ for the correct class. (bottom) shows the shift $\alpha_c$ for $A$ and $\Tilde{A}$ using scores $A'$ for the incorrect classes.}
    \label{fig:APS:efficiency}
\end{figure}

%\pmcomment{empirical quantiles with epsilon}

%\begin{align*}
%   |S(\vx_{n+1}, u; \hat{\pi}, \tau_A)| - |\Tilda{S}(\vx_{n+1}; \hat{\pi}, \tau_{})|
%\end{align*}

\subsection{Notes on Transductive NAPS}
The Neighborhood Adaptive Prediction Sets (NAPS) can construct predictive sets via Conformal Prediction under relaxed exchangeability (or non-exchangeability) assumptions \cite{barber2023NAPS}. In the context of graphs, NAPS was initially implemented in the inductive setting \cite{clarkson2023distribution}. However, it can be used in the transductive setting as well \cite{zargarbashi23conformal}. 


Transductive NAPS is based on APS where $s_i = A(\vx_i, y_i, u_i;\hat{\pi}_i)$, or $A(\vx_i, y_i;\hat{\pi}_i)$ if $\epsilon$ is not used, is computed for each node in $\gD_{\text{calib}}$. Using these scores, a weighted quantile is computed to produce a threshold for a class to be included in the prediction set as seen in Equation \ref{eq:NAPS:quantile} below by creating weighted point masses ($\delta$) at each score. The point mass at $+\infty$ indicates that the score for test node n+1 is unknown (and unbounded due to non-exchangeability), and thus, a point mass at the maximum value ($+\infty$) is required.\ascomment{I want to state the intuitive reason for the infinite point mass since it was not clearly stated in the NAPS paper.} 
\begin{align}
    \hat{q}^{\text{NAPS}}_{n+1} = \text{Quantile}\bigg(1-\alpha, \bigg[\sum_{i\in\gD_{\text{calib}}}\Tilde{w}_{i}\cdot \delta_{s_{i}}\bigg] + \Tilde{w}_{n+1}\cdot \delta_{+\infty}\bigg)
    \label{eq:NAPS:quantile}
\end{align}

Since exchangeability is not assumed, a weight function leveraging graph homophily can be used to produce weights, $w_i\in [0,1]$, for nodes in the calibration set \cite{barber2023NAPS}. The three implemented weight functions are uniform, $w_u(d_i) = 1$, hyperbolic $w_h(d_i) = \frac{1}{d_i}$, and exponential, $w_e(d_i) = 2^{-d_i}$ for nodes in the $k$ hop neighborhood of test node n+1, $\mathcal{N}_{n+1}^k$, where $d_i$ is the distance from the test node to node $i$ in the calibration set. Formally, the weight function - in the transductive setting - for each node, $i\in\gD_{\text{calib}}$ can be seen in Equation \ref{eq:NAPS:weight} below, where $w^x(d_i)$ is the selected weight function. These weights are then linearly normalized to compute $\Tilde{w}_i$ such that $1 = \sum_{i\in\gD_{\text{calib}}} \Tilde{w}_i + \Tilde{w}_{n+1}$ \cite{barber2023NAPS}.
\begin{align}
    w_i = \begin{cases}
w_x(d_i), & i\in \gD_{\text{calib}}\cap\mathcal{N}_{n+1}^k\\
0,& i\in \gD_{\text{calib}}\setminus\mathcal{N}_{n+1}^k
\end{cases}
    \label{eq:NAPS:weight}
\end{align}

Using the NAPS quantile, $\hat{q}^{\text{NAPS}}_{n+1}$, the prediction sets can be constructed similarly to other Conformal Prediction algorithms. Lastly, it should be noted that transductive and inductive NAPS differ from each other since transductive NAPS applies non-zero weights only to a subset of calibration nodes, while inductive NAPS uses the entire k-hop neighborhood of a test node.\ascomment{Verify this claim (i.e. reread the paper)}  

\subsubsection{A Word on Implementation}
NAPS is computationally expensive with regard to time and memory. To redress, we took a batched approach that worked with sparse tensors as detailed in Algorithm \ref{alg:NAPS:Quantile}. The test nodes are first split up into batches. Then, for each batch, the distance to each node in the k-hop neighborhood is computed and returned. Then, the corresponding weights are computed before computing each test node's quantile. The batched approach ensures that enough memory is available for the necessary computations - especially for computing the distance to each node in the k-hop neighborhood.  

\begin{algorithm}
\caption{NAPS Quantile Implementation}\label{alg:NAPS:Quantile}
\begin{algorithmic}[1]
\Procedure{NAPS\_Quantile}{$w,k,\calib,\test,\mathcal{D},\mathcal{S}_{\text{calib}},b,\alpha$}
    \State $\{\mathcal{B}_1,\mathcal{B}_2,\hdots,\mathcal{B}_b\}\gets \Call{\text{split}}{\test,b}$ \Comment{Split test nodes into b batches}
    \State $q\gets \Call{\text{zeros}}{\test,1}$\Comment{$q\in \mathbb{R}^{|\test| \times 1}$}
    \For{$\mathcal{B}_n \in \{\mathcal{B}_1,\mathcal{B}_2,\hdots,\mathcal{B}_b\}$}
        \State $\text{k\_hop} \gets \Call{\text{Sparse\_k\_hop}}{k,\mathcal{B}_n,\calib,\mathcal{D}}$\Comment{k\_hop $\in \mathbb{R}^{|\mathcal{B}_n|\times|\calib|}$}
        \State $\text{weights}\gets \Call{\text{compute\_weights}}{w,\text{k\_hop}}$\Comment{weights $\in \mathbb{R}^{|\mathcal{B}_n|\times|\calib|}$}
        \State $q[\mathcal{B}_n]\gets \Call{\text{compute\_quantile}}{1-\alpha,\text{weights},\mathcal{S}_{\text{calib}}}$
    \EndFor\label{NAPSquantileendwhile}
    \State \textbf{return} $q$\Comment{Return the quantiles for each test node}
\EndProcedure
\end{algorithmic}
\end{algorithm}

To ensure scalability for large graphs, all of the computations up until computing the quantile were done via sparse tensors. Algorithm \ref{alg:NAPS:SparseKHop} illustrates how the distance to each calibration node in the k-hop neighborhood can be computed via supported sparse tensor operations in PyTorch. To ensure that the minimum distance to nodes in the k-hop neighborhood is reported, the sign function is applied to the matrix containing paths exactly n hops away. This is subtracted from the matrix containing distances up to n-1 hops. A value can only be less than 0 after this subtraction if the corresponding index in the matrix containing distances up to n-1 hops was 0. Using this, the nodes that are at a minimum n hops away can be identified and added to the matrix containing distances to calibration nodes in the k-hop neighborhood. The described calculations can be done using sgn, and signbit (to check if an element is negative) in PyTorch.\ascomment{Shorted this paragraph}  

\begin{algorithm}
\caption{Sparse K Hop Neighborhood Implementation}\label{alg:NAPS:SparseKHop}
\begin{algorithmic}[1]
\Procedure{Sparse\_k\_hop}{$k,\mathcal{B},\calib,\mathcal{D}$}
    \State $\text{A}\gets \Call{\text{Get\_Adjacency}}{\mathcal{D}}$ \Comment{Adjacency of $\mathcal{D}$, A $\in \mathbb{R}^{|\mathcal{D}|\times|\mathcal{D}|}$}
    \State $\text{path\_n}\gets \text{A[$\mathcal{B},:$]}$\Comment{path\_n $\in \mathbb{R}^{|\mathcal{B}|\times|\mathcal{D}|}$ }
    \State $\text{k\_hop}\gets \text{path\_n[$:,\calib$]}$\Comment{k\_hop $\in \mathbb{R}^{|\mathcal{B}|\times|\calib|}$ }
    \For{$n \in \{2,3,\hdots,k\}$}
        \State $\text{path\_n} \gets (\text{path\_n})\text{A}$
        \State $\text{neg\_if\_n}\gets \text{k\_hop}-\Call{\text{sgn}}{\text{path\_n[$:,\calib$]}}$\Comment{negative value $\implies$ n hops away}
        \State $\text{in\_n\_hop}\gets (\text{neg\_if\_n}<0)\times n$ \Comment{Nodes that are a min distance of n}
        \State $\text{k\_hop} \gets \text{k\_hop} + \text{in\_n\_hop}$
    \EndFor\label{khopendwhile}
    \State \textbf{return} $\text{k\_hop}$\Comment{$\forall_{i,j}\, \text{\textbf{If} dist($i,j$)}  \leq k \text{ \textbf{then} k\_hop[$i,j$]} = \text{dist}(i,j), \text{\textbf{else} k\_hop[$i,j$]}=0$}
\EndProcedure
\end{algorithmic}
\end{algorithm}

\subsection{Diffusion Adaptive Prediction Sets}
The Diffusion Adpative Prediction Sets (DAPS) approach for conformal node classification on graphs was introduced by \citet{zargarbashi23conformal}.
The intuition behind DAPS due to the prevalence of homophily in graphs, the non-conformity scores for two connected scores should be related.
DAPS uses a diffusion step to capture this relationship and uses the non-conformity scores modified by diffusion to generate the prediction sets.
Formally, suppose $s(v, y)$ is a point wise non-conformity score for a node $v$ and label $y$ (e.g., TPS or APS)
\[
    \hat{s}(v, y) = (1 - \lambda) s(v, y) + \frac{\lambda}{|
    \gN_v|} \sum\limits_{u \in \gN_v} s(u, y)
\]
where $\gN_v$ is the 1-hop neighborhood of $v$ and $\lambda \in [0, 1]$ is a hyperparameter controlling the diffusion.

\citet{zargarbashi23conformal} use the APS score as the point wise score in diffusion process as it is adaptive and uniformly distribution in $[0, 1]$ under oracle probability.
However, as we noted earlier, using class wise thresholds provides a mechanism to produce adaptive scores from TPS as well.
Thus, we create DTPS, a variation of DAPS using TPS scores as the point wise scores in the diffusion process.

\subsection{Conformalized GNN}
\begin{figure}
    \centering
    \includegraphics[width=\linewidth]{graphConformal/figures/CFGNN.pdf}
    \caption{Procedure for training CF-GNN. First (left), the base model is trained on the training set. Then, (middle) the CF-GNN is trained to maximize efficiency over the calibration set. Finally , (right) the non-conformity scores from the combined models are used to generate the prediction sets.}
    \label{fig:conformalized_gnn}
\end{figure}

Conformalized GNN (CF-GNN)~\citep{huang2024uncertainty} is a GNN based approach for conformal prediction.
The authors observe that inefficiencies are correlated between nodes having similar neighborhood topology in a graph setting.
They use a GNN during the calibration phase which is trained to correct the scores output from the base model to maximize the efficiency of the conformal prediction.
For classification-based losses, CF-GNN utilizes the fact that all steps in the conformal prediction stage for computing the prediction sets (non-conformity score computation, quantile computation, thresholding) can be expressed as differentiable operations.
Thus, a GNN can be trained directly using efficiency as a loss function.
Figure~\ref{fig:conformalized_gnn} provides a high-level overview of the CF-GNN approach.

\noindent \textbf{CF-GNN Implementation Improvements}
The choice of the conformal loss during calibration and test plays an important role in determining the overall performance of the CF-GNN.
\citet{huang2024uncertainty} use a TPS loss for the calibration phase and the non-randomized APS loss for constructing the final prediction sets.
Our preliminary experiments (Figure~\ref{fig:CFGNN:preliminary}) with replacing the APS loss with a randomized version demonstrated that these losses must be tuned carefully to ensure that the CF-GNN is able to improve upon the base models non-conformity scores.
Some improvements shown in CF-GNN (Figure~\ref{fig:CFGNN:preliminary}, right) get nullified when the randomized APS loss is used (left).
Additionally, CF-GNN uses full batch training which makes it unable to scale for larger graphs.
We implemented a batched version of CF-GNN to ensure that it can be used for larger graphs.
Finally, to speed up computation, we allow the use of cached outputs from the base model rather than having to sample neighbors for both the base model and the CF-GNN. 
This significantly speeds up the computation in both training and evaluation for CF-GNN.

\begin{figure}
    \centering
    \includegraphics[width=0.8\linewidth]{graphConformal/figures/PubMed_CF.png}
    %\includegraphics[width=0.7\linewidth]{graphConformal/figures/DBLP_CF.png}
    \caption{Comparing the efficiency (average output set size) for the base model and the CF-GNN on the Pubmed dataset. The plot on the left uses the fixed version of the APS score (with randomized sets) while on the right uses the non-randomized version.}
    \label{fig:CFGNN:preliminary}
\end{figure}



\section{Evaluation of Graph Conformal Prediction}
\subsection{Datasets}

\begin{table}
    \centering
    \begin{tabular}{ccccc}
        \toprule
        Dataset & Nodes & Edges & Classes & Features \\
        \midrule
        CiteSeer & 3,327 & 9,228 & 6 & 3,703 \\ 
        Amazon\_Photos & 7,650 & 238,163 & 8 & 745 \\
        Cora & 19,793 & 126,842 & 70 & 8,710 \\
        PubMed & 19,717 & 88,651 & 3 & 500 \\
        Coauthor\_CS &  18,333 & 163,788 & 15 & 6,805 \\
        Coauthor\_Physics & 34,493 & 495,924 & 5 & 8,415 \\
        \bottomrule
    \end{tabular}
    \caption{Summary statistics for Datasets chosen for evaluation.}
    \label{tab:conformal:datasets}
\end{table}


We selected datasets of varying sizes to evaluate the performance of the graph conformal prediction methods.
For the citation datasets, the nodes are publications, and the edges denote citation relationships.
Features are bag-of-words representations of the documents.
The task is to predict the category of each publication.
\textbf{CiteSeer} is a citation network dataset designed for the node classification task, with nodes as publications and edges denoting citation relationships.
\textbf{Amazon\_Photos} is a segment of the Amazon co-purchase graph~\cite{mcauley2015image} where nodes represent goods, edges represent goods frequently bought together, features are bag-of-words representations of product reviews, and the task is to predict the category of each good.
\textbf{Cora} We use CoraFull~\cite{shchur2018pitfalls}, an extended version of the common Cora citation network dataset.
The objective is to predict the category of each node (publication).
\textbf{PubMed} is a citation network dataset designed for the node classification task, with nodes as publications and edges denoting citation relationships. The goal is to predict the category of each node (publication).
\textbf{Coauthor\_CS} and \textbf{Coauthor\_Physics} are co-authorship graphs extracted from the Microsoft Academic Graph and used for KDD Cup 2016. In this dataset, nodes are authors and edges denoting co-authorship relationships. The task is to predict the most active field of study for each author.
Summary statistics for the datasets are provided in Table~\ref{tab:conformal:datasets}.
For all chosen datasets, we used the version provided by the Deep Graph Library~\cite{wang2019dgl}.
To help characterize the behavior of different approaches, we categorize these into sizes based on the number of nodes, with \textbf{CireSeer} and \textbf{Amazon\_Photos} designated as small (S), \textbf{Cora}, \textbf{PubMed}, and \textbf{Coauthor\_CS} as medium (M), and \textbf{Coauthor\_Physics} as large (L).

\subsection{Metrics}
We evaluate the following metrics for the graph conformal prediction methods:
\begin{itemize}
    \item \textbf{Coverage:} The proportion of test instances for which the true label is contained in the prediction set.
    \item \textbf{Efficiency:} The average size of the prediction set.
    \item \textbf{Label Stratified Coverage:} The mean of coverage for each class. This metric is useful for understanding whether a method is adaptive and has balanced coverage for different classes.
    \item \textbf{Size Stratified Coverage:} The mean of coverage across different sizes of prediction sets. This metric is useful for understanding whether a method is adaptive and does not under/over cover hard/easy samples.
    %\item \textbf{Size Stratified Coverage Violation:} Measures the maximum deviation from the coverage goal across different prediction set sizes. 
    %\item \textbf{Singleton Hit Ratio:} The proportion of test instances for which the true label is the only label in the prediction set. This metric measures the frequency with which a method does not require generation of a prediction set having multiple labels.
\end{itemize}

\subsection{Methods}
We discussed the theoretical and empirical tradeoffs of different methods in Section~\ref{chp:graphConformal:sec:conformal_scores_tradeoffs}.
For completeness, we list all the methods that we compare here.
\textbf{Threshold Prediction Sets}~\cite{sadinle2019least}, with two variants, TPS and TPS-Classwise (using class wise thresholds for adapting to class imbalance).
\textbf{Adaptive Prediction Sets}~\cite{romano2020classification} with two variants, APS and APS-Randomized (using the uniform random quantile adjustments).
\textbf{Regularized Adaptive Prediction Sets}~\cite{angelopoulos2021uncertainty}, a variation of APS with a regularization term to ensure that the prediction sets are not too large.
\textbf{Diffused Adaptive Prediction Sets}~\cite{zargarbashi23conformal}, with two variations DAPS and DTPS, which uses a diffusion process over TPS-Classwise.
\textbf{Normalized Adaptive Prediction Sets}~\cite{clarkson2023distribution} with three variations corresponding to the weighing function used.
\textbf{CF-GNN}~\cite{huang2024uncertainty}, a GNN based approach for conformal prediction. We label the original implementation of CFGNN as CFGNN-Original and our improved implementations as CFGNN-APS (using randomized APS as the loss function for training/evaluation) and CFGNN-TPS (using TPS as the loss function for training/evaluation).

%\subsection{Notes on Parameter Tuning and Evaluation}
%\pmcomment{TODO for final version}

\section{Results}


\begin{figure}
    \centering
    \includegraphics[width=\linewidth,alt={Line plots showing method comparisons for small datasets.}]{graphConformal/figures/split/small_datasets_efficiency.png}
    \includegraphics[width=\linewidth,alt={Line plots showing method comparisons for Medium datasets.}]{graphConformal/figures/split/med_1_datasets_efficiency.png}
    \includegraphics[width=\linewidth,alt={Line plots showing method comparisons for large datasets.}]{graphConformal/figures/split/med_2_datasets_efficiency.png}
    \caption{Plots for efficiency vs $\alpha$ for the major methods across the all the datasets. Among the baseline methods, TPS consistently has the best efficiency. Result for FS paritttion}
    \label{fig:fs:conformal:efficiency_vs_alpha}
\end{figure}

First, we analyze the efficiency of the methods across different datasets.
Figure~\ref{fig:fs:conformal:efficiency_vs_alpha} shows the efficiency of the methods across different datasets.
We find that for each dataset, irrespective of the train/validation/calib split, TPS is consistently the most efficient method.
However, this often comes at a cost to adaptability.
In the next set of results, we show how using classwise thresholds can provide some degree of adaptability for TPS.
Next, we focus on the adaptability provided by using classwise TPS.

\subsection{Adaptability through Classwise TPS}
\begin{figure}
    \centering
    \begin{subfigure}{0.48\linewidth}
    \includegraphics[width=\linewidth,alt={Box plots for method comparison on CiteSeer dataset label stratified coverage.}]{graphConformal/figures/split/citeseer_label_stratified_coverage.png}
    \end{subfigure}
    \begin{subfigure}{0.48\linewidth}
        \includegraphics[width=\linewidth,alt={Box plots for method comparison on CiteSeer dataset size stratified coverage.}]{graphConformal/figures/split/citeseer_size_stratified_coverage.png} 
     \end{subfigure}

    \begin{subfigure}{0.48\linewidth}
    \includegraphics[width=\linewidth,alt={Box plots for method comparison on Cora dataset label stratified coverage.}]{graphConformal/figures/split/cora_label_stratified_coverage.png}
    \end{subfigure}
    \begin{subfigure}{0.48\linewidth}
        \includegraphics[width=\linewidth,alt={Box plots for method comparison on Cora dataset size stratified coverage.}]{graphConformal/figures/split/cora_size_stratified_coverage.png} 
     \end{subfigure}
     \caption{At a target $\alpha = 0.1$. Boxplots indicating (left) Label Stratified Coverage. (right) Size Stratified Coverage for CiteSeer (top) and Cora(bottom). Classwise TPS provides adaptability when stratified by labels without sacrificing size stratified coverage. Results for FS splits.}
     \label{fig:fs:conformal:citeseer_adaptability}
\end{figure}
From Figure~\ref{fig:fs:conformal:citeseer_adaptability}, we see that using classwise TPS successfully provides stratified coverage over different labels without sacrificing size stratified coverage vs a baseline TPS.
At the limit, even when reducing the number of samples per class from Figure~\ref{fig:nspc:citeseer:ssc}, we can see that the loss in size stratified coverage is minimal.
Thus, at least for the datasets we studied, TPS-Classwise is a good candidate for an adaptive version of TPS.
\begin{figure}
    \begin{subfigure}{0.48\linewidth}
        \includegraphics[width=\linewidth,alt={Box plots for method comparison on Amazon Photos dataset size stratified coverage with 10 samples per class.}]{graphConformal/figures/nspc/amazon_photos_10_size_stratified_coverage.png}
    \end{subfigure}
    \begin{subfigure}{0.48\linewidth}
        \includegraphics[width=\linewidth,alt={Box plots for method comparison on Amazon Photos dataset size stratified coverage with 40 samples per class.}]{graphConformal/figures/nspc/amazon_photos_40_size_stratified_coverage.png}
    \end{subfigure}
    \caption{At a target $\alpha = 0.1$, boxplots for size stratified coverage with calibration sets having (left) 10 samples per class and (right) 40 samples per class for Amazon Photos.}
    \label{fig:nspc:citeseer:ssc}
\end{figure}

\subsection{APS Randomized Sets}
\begin{figure}
    \centering
    \begin{subfigure}{0.8\linewidth}
    \includegraphics[width=\linewidth,alt={Violin plots comparison randomized and non randomized efficieincy for APS across 5 datasets.}]{graphConformal/figures/split/aps_randomized_efficiency.png}
    \end{subfigure}
    \begin{subfigure}{0.6\linewidth}
        \includegraphics[width=\linewidth,alt={Violin plots comparison randomized and non randomized efficieincy for APS on Cora.}]{graphConformal/figures/split/aps_randomized_efficiency_cora.png}
    \end{subfigure}
    \caption{Violin plots denoting efficiencies of APS and Randomized APS across different datasets and multiple runs in FS split. Randomization consistently improves over the non-randomized version.}
    \label{fig:fs:conformal:aps_vs_randomized}
\end{figure}

\begin{comment}
\begin{figure}
    \centering
    \begin{subfigure}{0.8\linewidth}
    \includegraphics[width=\linewidth]{graphConformal/figures/nspc/aps_randomized_efficiency}
    \end{subfigure}
    \begin{subfigure}{0.6\linewidth}
        \includegraphics[width=\linewidth]{graphConformal/figures/nspc/aps_randomized_efficiency_cora}
    \end{subfigure}
    \caption{Violin plots denoting efficiencies of APS and Randomized APS across different datasets and multiple runs in LC split. Randomization consistently improves over the non-randomized version.}
    \label{fig:nspc:conformal:aps_vs_randomized}
\end{figure}
\end{comment}

Figure~\ref{fig:fs:conformal:aps_vs_randomized} %and Figure~\ref{fig:nspc:conformal:aps_vs_randomized} 
provide violin plots that compare the efficiency of randomized and non-randomized version of APS across different datasets and $\alpha$.
We observe that in each case, the peaks associated with the randomized version lie to the left of those associated with the non-randomized version.
This indicates that the randomized version consistently provides a more efficient prediction set.
This effect is most pronounced for a dataset having a large number of potential classes (Cora), which matches with the intuition from Theorem~\ref{them:APS:efficiency} - with a $(K-1)\left(\alpha_c^{A} - \alpha_C^{\Tilde{A}}\right)$ term contributing to the improved efficiency and least pronounced for PubMed, which has the smallest $K=3$.
Overall, the empirical results show that the effect of randomized APS is more apparent for larger number of classes $K$.
% \pmcomment{Show that TPS-classwise does not sacrifice much on efficiency either}

\subsection{Diffusion Thresholded Adaptative Sets}
\begin{figure}
    \centering
    \begin{subfigure}{0.7\linewidth}
        \includegraphics[width=\linewidth,alt={Bar charts denoting different metrics associated with DAPS and DTPS across PubMed }]{graphConformal/figures/split/daps_dtps_pubmed.png}
    \end{subfigure}
    \begin{subfigure}{0.7\linewidth}
        \includegraphics[width=\linewidth,alt={ar charts denoting different metrics associated with DAPS and DTPS across Cora.}]{graphConformal/figures/split/daps_dtps_cora.png}
    \end{subfigure}
    \caption{Bar charts denoting different metrics associated with DAPS and DTPS across PubMed (top) and Cora (bottom) for the TS split at $\alpha=0.1$. We see that DTPS improves efficiency for PubMed but not for Cora, with minimal impact to other adaptive metrics.}
    \label{fig:fs:conformal:daps_vs_dtps}
\end{figure}


\begin{figure}
    \centering
    \begin{subfigure}{0.7\linewidth}
        \includegraphics[width=\linewidth,alt={Bar charts denoting different metrics associated with DAPS and DTPS across the LC split at $\alpha=0.1$}]{graphConformal/figures/nspc/daps_dtps_0.1.png}
    \end{subfigure}
    \begin{subfigure}{0.7\linewidth}
        \includegraphics[width=\linewidth,alt={Bar charts denoting different metrics associated with DAPS and DTPS across the LC split at $\alpha=0.2$}]{graphConformal/figures/nspc/daps_dtps_0.2.png}
    \end{subfigure}
    \caption{Bar charts denoting different metrics associated with DAPS and DTPS across the LC splits at $\alpha=0.1$ (top) and $\alpha=0.2$ (bottom). We see that DTPS deteriorates significantly as compared to DAPS at higher $\alpha$.}
    \label{fig:nspc:conformal:daps_vs_dtps}
\end{figure}

We compare our proposed Diffusion method--of using TPS-Classwise as the base method--DTPS against DAPS, which was proposed in~\cite{zargarbashi23conformal}.
From Figure~\ref{fig:fs:conformal:daps_vs_dtps}, we see that when the calibration set is large (TS), DTPS can improve efficiency without sacrificing adaptiveness for PubMed but not for Cora.
However, when we control the number of samples per class with LC splits (Figure~\ref{fig:nspc:conformal:daps_vs_dtps}), we see that DTPS deteriorates significantly as compared to DAPS at higher $\alpha$.
Based on these results, we can conclude that DTPS is not a universally better method than DAPS, and its performance is sensitive to the calibration set size and the number of classes.
It may be a viable candidate over DAPS when there is a sufficiently large calibration set.

\subsection{CFGNN}
We first describe the runtime improvements achieved by using batching and caching in our CFGNN implementation, and follow it up with an evaluation of CFGNN-APS (randomized) and CFGNN-Original on the FS and LC splits.
\subsubsection{Runtime}
\label{sec:conformal:results:cfgnn:runtime}

\begin{table}[h]
    \centering
    \begin{tabular}{llll}
        \toprule
        method & baseline & batching & cache+batch \\
        dataset &  &  &  \\
        \midrule
        CiteSeer & 186.61 $\pm$ 11.43 & 8.64 $\pm$ 1.76 & 4.43 $\pm$ 0.13 \\
        Amazon\_Photos & 291.11 $\pm$ 6.20 & 29.66 $\pm$ 1.03 & 8.27 $\pm$ 0.18 \\
        Cora & 985.99 $\pm$ 62.42 & 89.80 $\pm$ 1.50 & 28.82 $\pm$ 2.08 \\
        PubMed & 254.38 $\pm$ 8.09 & 58.26 $\pm$ 1.68 & 15.31 $\pm$ 0.63 \\
        Coauthor\_CS & 669.48 $\pm$ 34.75 & 72.97 $\pm$ 1.21 & 17.70 $\pm$ 1.64 \\
        Coauthor\_Physics & 2089.23 $\pm$ 80.00 & 758.22 $\pm$ 15.28 & 27.63 $\pm$ 0.81 \\
        \bottomrule
    \end{tabular}
    \caption{Runtime for CFGNN implementations starting from the baseline, then adding batching, and then adding caching and batching combined. For each setup we compare the results from 5 runs and provide 95\% confidence intervals in the reported results. All runtimes in seconds, runs executed on a single A100 GPU.}
    \label{tab:conformal:cfgnn_runtime}
\end{table}

We compare three variations of the CFGNN implementation to demonstrate the impact of batching and caching on the runtime.
Across all comparisons, we use the FS split, with 20\%/20\% assigned to train/valid sets, and 35\% to the calibration dataset.
For ease of comparison, we fix the CFGNN architecture to a 2-layer GCN having 128 hidden units.
We use the best base GNN parameters for each dataset and split.
The baseline implementation follows the setup used by \citep{huang2024uncertainty}, where the CFGNN is trained with full batch gradient descent for 1000 epochs.
Our improved implementation, which uses batched descent, is able to achieve an equivalent efficiency in only 20 epochs, without any batch size tuning (we set the batch size to 64 for consistent comparison).
Finally, we add caching of the output probabilities from the base GNN to the batched implementation, which further reduces the runtime.
Table~\ref{tab:conformal:cfgnn_runtime} describes the comparison of the batching, and the combined batching + caching improvements.
We discard the first run in each experiment as it includes the warm up time for running on the GPU.
We note that our implementation is able to achieve improvements ranging from 16.6x (PubMed) to 75.6x (Coauthor\_Physics) in runtime over the baseline implementation.

\subsubsection{Evaluation}
\begin{figure}
    \centering
    \begin{subfigure}{0.7\linewidth}
        \includegraphics[width=\linewidth,alt={Bar charts denoting efficiency for CFGNN-APS and CFGNN-Original across the TS split at $\alpha=0.1$. We see that CFGNN-APS improves or matches efficiency in most cases.}]{graphConformal/figures/split/cfgnn_aps_vs_orig_efficiency.png}
    \end{subfigure}
    \caption{Bar charts denoting efficiency for CFGNN-APS and CFGNN-Original across the TS split at $\alpha=0.1$. We see that CFGNN-APS improves or matches efficiency in most cases.}
    \label{fig:fs:conformal:cfgnn_aps_vs_orig}
\end{figure}

\begin{figure}
    \centering
    \begin{subfigure}{0.48\linewidth}
        \includegraphics[width=\linewidth,alt={Bar charts denoting efficiency for CFGNN-APS and CFGNN-Original across the LC split at $\alpha=0.1$ with 10 samples per  class.}]{graphConformal/figures/nspc/cfgnn_aps_vs_orig_efficiency_10.png}
    \end{subfigure}
    \begin{subfigure}{0.48\linewidth}
        \includegraphics[width=\linewidth, alt={Bar charts denoting efficiency for CFGNN-APS and CFGNN-Original across the LC split at $\alpha=0.1$ with 20 samples per  class.}]{graphConformal/figures/nspc/cfgnn_aps_vs_orig_efficiency_20.png}
    \end{subfigure}
    \caption{Bar charts denoting efficiency for CFGNN-APS and CFGNN-Original across the LC split at $\alpha=0.1$ with 10 samples per  class (left) and 20 samples per class (right). We see that CFGNN is unstable for the LC setting.}
    \label{fig:nspc:conformal:cfgnn_aps_vs_orig}
\end{figure}

We implement an improved version of CFGNN-APS which uses the randomized APS loss in both training and evaluation.
In contrast, CFGNN-Original uses TPS during training and non-randomized APS during evaluation.
We compare the efficiency of CFGNN-APS and CFGNN-Original in Figure~\ref{fig:fs:conformal:cfgnn_aps_vs_orig}.
We see that CFGNN-APS improves or matches efficiency in 5/6 cases.
For these results, we only trained the parameters of the CFGNN, keeping the architecture fixed.
Further tuning of the architecture may improve the performance of CFGNN-APS.

Finally, originally, CFGNN was evaluated on FC splits.
We benchmark its performance on LC splits in Figure~\ref{fig:nspc:conformal:cfgnn_aps_vs_orig}.
We see that CFGNN is unstable for the LC setting.
One potential reason for this is that the CFGNN is not designed to handle the LC setting as the data may be insufficient to train a conformal model.
Exploring methods to improve the stability of CFGNN in the LC setting is an area for future work.
%This is an area for future work.

\section{Conclusion}
In this chapter, we demonstrated the tradeoffs associated with the choices made in the implementation of graph conformal prediction.
We provide various recommendations for different dataset splits, methods, and evaluation metrics, which indicates relevant directions for future work in graph conformal prediction.

\begin{subappendices}
    \section{Optimal $\tau$ for APS}
\label{appx:APS:tau}
For simplicity, assume that the probabilities are distinct.

From the definition of $A$ \eqref{eq:APS:score}
\begin{align*}
    A(\vx, y, u;\hat{\pi}) &= \min\{\tau \in [0, 1]: y \in S(\vx, u; \hat{\pi}, \tau)\}
\end{align*}
Define 
\[
\Sigma_{\hat{\pi}}(\vx, m) = \sum\limits_{i=1}^m \hat{\pi}_{(i)}(\vx)
\]
From the definition of $S(\vx, u; \hat{\pi}, \tau)$ from \eqref{eq:APS:S}, conisder the following cases:

\textbf{Case 1:} $\tau = \Sigma_{\hat{\pi}}(\vx, r_y)$, then $L(x; \hat{\pi}, \tau) = y$ and thus, $V(\vx; \pi, \tau) = 0$.
Thus $\Pr[u > V(\vx; \pi, \tau)] = 1$ and hence, $P[y \in S(\vx, u; \hat{\pi}, \tau)] = 1$.

\textbf{Case 2:} $\tau = \Sigma_{\hat{\pi}}(\vx, r_y-1)$, then $y \not\in S(\vx, u, \hat{\pi}, \tau)$ in either case, since only classes with $\hat{\pi}_i(\vx) > \hat{\pi}_y(\vx)$ could be included. \pmcomment{This is the edge case where tie breaking is required for a completely general proof.}

\textbf{Case 3:} $\tau = \Sigma_{\hat{\pi}}(\vx, r_y) - \varepsilon \hat{\pi}_y$.
Then we have $L(x; \hat{\pi}, \tau) = y$ again, and 
\begin{align*}
    V(\vx; \pi, \tau) &= \frac{1}{\hat{\pi}_y(\vx)}\left\{ \left[ \sum_{j=1}^{r_y} \hat{\pi}_{(j)}(\vx) \right] - \tau \right\} \\
                      &= \frac{1}{\hat{\pi}_y(\vx)}\left\{ \left[ \sum_{j=1}^{r_y} \hat{\pi}_{(j)}(\vx) \right] - (\Sigma_{\hat{\pi}}(\vx, r_y) - \varepsilon \hat{\pi}_y) \right\}\\
                      &= \varepsilon
\end{align*}
For $y$ to be included in $S(\vx, u; \hat{\pi}, \tau)$, we would require that $u \geq V(\vx; \pi, \tau)$, i.e., $u \geq \varepsilon$. 
We want the minimal $\tau$, which is equivalent to maximizing $\varepsilon$. 
Thus, $\tau = \Sigma_{\hat{\pi}}(\vx, r_y) - u \hat{\pi}_y$ is the required solution.

\subsection{Non-randomized set}
The inclusion criterion for the score given the threshold $\tau$ is $\Tilde{A}(\vx, y; \hat{pi}) \leq \tau$

To include the currct label $y_i$ while minimizing the chosen threshold $\tau$, we would require $\tau = \sum\limits_{j=1}^{r_{y_i}} \hat{\pi}_{(j)}(\vx)$ 
\end{subappendices}
